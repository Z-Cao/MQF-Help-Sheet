\documentclass[12pt,landscape, a4paper]{article}
\usepackage{multicol}
\usepackage{calc}
\usepackage{ifthen}
\usepackage[landscape]{geometry}
\usepackage{hyperref}
\usepackage[english]{babel}
\usepackage{amsthm}
\usepackage{amsmath}
\usepackage{amsfonts}
\usepackage{amssymb}
\usepackage{graphicx}
\usepackage{enumitem, kantlipsum}
\usepackage[nopar]{lipsum}
\usepackage[T1]{fontenc}
\usepackage{dsfont}
% \usepackage{unicode-math}
% \setmathfont{XITS Math}
% \usepackage{fontspec}
% \setmainfont{TeX Gyre Pagella}
\usepackage[scr=boondox]{mathalpha}
\usepackage{pxfonts}

\geometry{top=0.45cm,left=0.45cm,right=0.45cm,bottom=0.45cm}

% Turn off header and footer
\pagestyle{empty}
 

% Redefine section commands to use less space
\makeatletter
\renewcommand{\section}{\@startsection{section}{1}{0mm}%
                                {-0.5ex plus -.2ex minus -.2ex}%
                                {0.2ex plus .2ex}%x
                                {\small\small\bfseries}}

\renewcommand{\subsection}{\@startsection{subsection}{1}{0mm}%
                                {-0.5ex plus -.2ex minus -.2ex}%
                                {0.2ex plus .2ex}%x
                                {\scriptsize\scriptsize\bfseries}}
                                
% Don't print section numbers
\setcounter{secnumdepth}{0}
\setlength{\parindent}{0pt}
\setlength{\parskip}{0pt plus 0.5ex}

\theoremstyle{remark}
\newtheorem*{thm}{Thm}
\newtheorem*{defn}{Def}
\newtheorem*{lemma}{Lemma}
\newtheorem*{corollary}{Corollary}
\newtheorem*{question}{Question}
\newenvironment{soln}{\begin{proof}[Solution]}{\end{proof}}

\newcommand{\var}{\operatorname{Var}}
\newcommand{\E}{\operatorname{\mathbb{E}}}
\newcommand{\prob}{\operatorname{\mathbb{P}}}
\newcommand{\cov}{\operatorname{Cov}}
\newcommand{\R}{\mathcal{R}}
\newcommand{\VaR}{\mathrm{VaR}}
\newcommand{\ES}{\mathrm{ES}}
\newcommand{\EV}{\mathrm{EV}}
\newcommand{\CF}{\mathrm{CF}}
\newcommand{\Dom}{\operatorname{Dom}}
\newcommand{\NAV}{\mathrm{NAV}}
\newcommand{\dd}{\partial}
\newcommand{\N}{\mathcal{N}}

\begin{document}

\setlength{\abovedisplayskip}{0pt}%
\setlength{\belowdisplayskip}{0pt}%
\setlength{\abovedisplayshortskip}{0pt}%
\setlength{\belowdisplayshortskip}{0pt}%
\setlength{\jot}{0pt}% Inter-equation spacing

\fontfamily{phv}

\raggedright
\tiny
%\scriptsize
\begin{multicols*}{3}
Put-Call Parity: $C - P = S_0 - Ke^{-rT}$
$\frac{d}{dx} \int_{a(x)}^{b(x)} f(x,t) \, dt = f(x, b(x)) \cdot b'(x) - f(x, a(x)) \cdot a'(x) + \int_{a(x)}^{b(x)} \frac{\partial}{\partial x} f(x,t) \, dt$\\

\textbf{VIX} Measure of stock market's expectation of volatility implied by SP500 index options\\
\textbf{Historical Vol} Assume $d S_t = S_t (\mu dt + \sigma d B_t)$ or $d \log S_t = \left(\mu - \frac{\sigma^2}{2} \right) dt + \sigma d B_t$\\
Denote $t_i = i \delta t, i=0, \dots, N$, and $S_i := S_{t_i}$, then we obtain the discretisation as follows,
\begin{align*}
    &\log \left(\frac{S_i}{S_{i-1}} \right) = \log S_i - \log S_{i-1} \approx \left(\mu -\frac{\sigma^2}{2}  \right) \delta t + \sigma \phi \sqrt{\delta t} ,\quad \phi \sim \N (0, 1), i = 1, \dots, N\\
    \Rightarrow& \E \left[ \left(\log \frac{S_i}{S_{i-1}} \right)^2 \right] \approx \sigma^2 \delta t \quad \Rightarrow \quad
    \sigma = \sqrt{\frac{1}{\delta t} \frac{1}{N} \sum^N_{i=1} \left(\log \frac{S_i}{S_{i-1}} \right)^2 }
\end{align*}
\textbf{Implied Vol} Consider a call option, $\sigma_{imp}$ is defined as $C_{MKT} (S_t, t; K, T) = S_t \N (d_1 (\sigma_{imp})) - K e^{-r (T-t) } \N (d_2 (\sigma_{imp})), d_{1, 2} (\sigma) = \frac{\log (S_t/K) + (r \pm \sigma^2/2) (T-t) }{\sigma \sqrt{T-t}}$\\
$\sigma_{imp}$ is well-defined since $\frac{\dd C}{\dd \sigma} > 0$\\

\textbf{Linear Interpolation} Between $(t_1, a), (t_2, b), f(t) = \frac{t_2 - t}{t_2 - t_1} a + \frac{t - t_1}{t_2 - t_1 } b$\\

\textbf{Old VIX (VXO)} Definition based on one-month implied volatility of at-the-money options on SP100
\begin{align*}
    \mathrm{OldVIX} = 100 \sqrt{\frac{365}{30} \left[T_1 V_{T_1} \frac{N_2 - 30}{N_2 - N_1} + T_2 V_{T_2} \frac{30 - N_1}{N_2 - N_1} \right] }
\end{align*}
where $T_i$ is time to maturity (in year); $N_1, N_2$ are number of days to expiration for two nearest maturities (larger than 8 days, in minutes); $V_T$ is average of two calls and two puts whose strikes are nearest to prevailing price of SP100
\begin{flalign*}
    &\textbf{New VIX } 100 \sqrt{\frac{365}{30} \left[T_1 V_{T_1} \frac{N_2 - 30}{N_2 - N_1} +T_2 V_{T_2} \frac{30 - N_1}{N_2 - N_1} \right] }, \quad V_T = \frac{2}{T} \sum_i \frac{\Delta K_i}{K_i^2} e^{rT} Q(K_i) - \frac{1}{T} \left(\frac{F}{K_0} - 1 \right)&
\end{flalign*}
$Q (K_i)$ is midpoint of bid-ask spread for each out-of-money option with strike $K_i$; $\Delta K_i = (K_{i+1} - K_{i-1}) / 2$, $\Delta K$ for lowest strike is difference between lowest strike and next higher strike; $F = e^{rT} [C (K) - P(K)] + K $ Forward SP500 index level ($F = S_0 e^{rT}$ if assuming const. $r$); $K_0 \leq F$ is first strike below $F$; $K_i$ is strike of $i$-th out-of-money option, call if $K_i > K_0$, put if $K_i <K_0$, both call and put if $K_i = K_0$\\

\textbf{Payoff of Var Swap}: notional amount $\times$ (RealisedVar $-$ FixedVar)\\
\textbf{Payoff of Vol Swap}: notional amount $\times$ (RealisedVol $-$ FixedVol)\\
RealisedVar $-$ FixedVar $=\frac{252}{N} \sum^N_{i=1} \left(\log \frac{S_i}{S_{i-1}} \right)^2 - X_{var}$\\
RealisedVol $-$ FixedVol $=\sqrt{\frac{252}{N} \sum^N_{i=1} \left(\log \frac{S_i}{S_{i-1}} \right)^2  } - X_{vol}$\\
Assume notional amount is $1$ without loss of generality\\
With $d \log S_t = \left(\mu_t - \sigma^2_t/2 \right) dt + \sigma_t d B_t$, $\frac{252}{N} \sum^N_{i=1} \left(\frac{S_i}{S_{i-1}} \right)^2 \approx \frac{1}{N \delta t} \sum^N_{i=1} \sigma^2_{t_{i-1}} \delta t \to \frac{1}{T} \int^T_{0} \sigma^2_t dt$ as $N\to \infty$\\

\textbf{Var Swap Rate} $e^{-rT} \hat{\E} \left[\frac{1}{T} \int^T_0 \sigma^2_t dt - X_{var} \right] = 0 \quad \Rightarrow \quad X_{var} = \hat{\E} \left[\frac{1}{T} \int^T_0 \sigma^2_t dt \right]$\\
Since $\int^T_0 d \log S_t = \int^T_0 \left(\mu_t - \sigma^2_t/2 \right) dt + \int^T_0 \sigma_t d B_t $, $ \int^T_0 \frac{\sigma^2_t}{2} dt = \int^T_0 r dt + \int^T_0 \sigma_t d \hat{B}_t - \int^T_0 d\log S_t$,\\
where $\int^T_0 r dt- \int^T_0 d\log S_t = rT - \log \left(\frac{S_T}{S_0} \right) = -\log \left(\frac{S_T}{S_0 e^{rT}} \right) =: -\log \left(\frac{S_T}{F_0} \right) $,\\
Thus, $\frac{1}{T} \int^T_0 \sigma^2_t dt = \frac{2}{T} \hat{\E} \left[-\log \left(\frac{S_T}{F_0} \right)  + \int^T_0 \sigma^t d \hat{B}_t \right] = -\frac{2}{T} \log \frac{S_T}{F_0}$, since $\int^T_0 \sigma^t d \hat{B}_t$ is martingale\\

\textbf{Decomposition} $-\log \frac{S_T}{F_0} = -\log \frac{S_T}{F_0} - \left( -\log \frac{F_0}{F_0} \right) = - \frac{S_T - F_0}{F_0} + \int^\infty_{F_0} \frac{1}{K^2} (S_T - K)^+ dK + \int^{F_0}_0 \frac{1}{K^2} (K - S_T)^+ dK$, i.e. $-\log \frac{S_T}{F_0} = -\text{forward} + \text{OTM Call} + \text{OTM put}$, implies var swap can be statically replicated by vanillas
\begin{align*}
    \text{Proof: } f(x) - f(a) &= f^\prime (a) (x-a) + \int^x_a f^{\prime \prime} (K) (x-K) dK 
    = f^\prime (a) (x-a) + \int^x_a f^{\prime \prime} \left\lvert (x - K)^+ - (K-x)^+ \right\rvert dK\\
    &= f^\prime (a) (x-a) + \int^x_a f^{\prime \prime} (x-K)^+ dK + \int^a_x f^{\prime \prime} (K-x)^+ dK
\end{align*}
\begin{flalign*}
    \text{Combining equations above, }X_{var} &= \frac{2}{T} \left[\int^{+\infty}_{F_0} \frac{1}{K^2} \hat{\E} (S_T - K)^+ dK + \int^{F_0}_0 \frac{1}{K_2} \hat{\E} (K - S_T)^+ dK  \right]&\\
    &= \frac{2}{T} \left[\int^{+\infty}_{F_0} \frac{1}{K^2} e^{rT} C_0 (K_T) dK + \int^{F_0}_0 \frac{1}{K^2} e^{rT} P_0 (K, T) dK \right]&
\end{flalign*}
\textbf{Linkage} $K_0 = F_0 \Rightarrow$ VIX is discrete version of variance swap rate; $K_0 \neq F_0$, then replace $F$ with $K_0$, $-\log \frac{S_T}{F_0} = -\log \frac{S_T}{F_0} - \left( -\log \frac{F_0}{F_0} \right) = - \frac{S_T - K_0}{K_0} + \int^\infty_{K_0} \frac{1}{K^2} (S_T - K)^+ dK + \int^{K_0}_0 \frac{1}{K^2} (K - S_T)^+ dK$, then
\begin{align*}
    X_{var} &= \frac{2}{T} \hat{\E} \left[-\log \frac{S_T}{F_0} \right] = \frac{2}{T} \hat{\E} \left[-\log \frac{S_T}{K_0} + \log \frac{F_0}{K_0} \right] \\
    &= \frac{2}{T} \hat{\E} \left[ -\frac{S_T - K_0}{K_0} + \log \frac{F_0}{K_0} \right] + \frac{2}{T} \left[ \int^{+\infty}_{K_0} \frac{1}{K^2} e^{rT} C_0 (K, T) dK + \int^{K_0}_{0} \frac{1}{K^2} P_0 (K, T)  \right]
\end{align*}
$\frac{2}{T} \hat{\E} \left[ -\frac{S_T - K_0}{K_0} + \log \frac{F_0}{K_0} \right] = -\frac{F_0 - K_0}{K-0} + \log \frac{F_0}{K_0 } = -\frac{F_0 - K_0}{K-0} + \left(\frac{F_0}{K_0} -1 \right) - \frac{1}{2} \left(\frac{F_0}{K_0} -1 \right)^2 + \mathcal{O} \left(\frac{F_0}{K_0} -1 \right)^3 $. \\
By omitting $ \mathcal{O} \left(\frac{F_0}{K_0} -1 \right)^3$, $\frac{2}{T} \hat{\E} \left[ -\frac{S_T - K_0}{K_0} + \log \frac{F_0}{K_0} \right] \approx -\frac{1}{T} \left(\frac{F_0}{K_0} -1 \right)^2$\\

\textbf{Pricing Vol Product} Consider $V = V(S_t, I_t, t), I_t = \int^t_0 \sigma_{loc} (S_\tau, \tau) d\tau, t\in [0, T), S>0, I>0$, \\
$\frac{\dd V}{\dd t} + \frac{1}{2}\sigma^2_{loc} (S, t) S^2 \frac{\dd^2}{\dd S^2} + rS \frac{\dd V}{\dd S} - rV + \sigma_{loc}^2 (S, t) \frac{\dd V}{\dd I} = 0$, with terminal condition $V\lvert_{t=T} = f(\frac{1}{T})$\\
\textbf{Derivation 1} Let $\Pi = V - \Delta S$, then $d \Pi = d V - \Delta d S = \left[\frac{\dd V}{\dd t} + \frac{1}{2} \sigma^2 S^2 \frac{\dd^2 V}{\dd S^2} \right] dt + \frac{\dd V}{\dd S} dS + \frac{\dd V}{\dd I} dI - \Delta dS$. Choose $\Delta = \frac{\dd V}{\dd S}$, and $d I = \sigma^2 dt$ is known\\
\textbf{Derivation 2} $dV = \left[\frac{\dd V}{\dd t} + \frac{1}{2} \sigma^2 S^2 \frac{\dd^2 V}{\dd S^2} \right] dt + \frac{\dd V}{\dd S} dS + \frac{\dd V}{\dd I}d I$, then substitute $dS, dI$
\begin{flalign*}
    &\textbf{Dupire }\frac{\dd V}{\dd T} -\frac{1}{2} \sigma^2_{loc} (K, T) K^2 \frac{\dd^2 V}{\dd K^2} + rK \frac{\dd V}{\dd K} = 0, \sigma^2_{loc} (K, T) = \frac{\frac{\dd V}{\dd T} + rK \frac{\dd V}{\dd K} }{\frac{1}{2}K^2 \frac{\dd^2 V}{\dd K^2}}&
\end{flalign*}
\textbf{Proof} Let $\phi (S, t; x, T)$ be conditional pdf at $T$ given $(S, t)$, then consider option pricing,\\
$C (S, t, K, T) = \exp \{-r (T-t) \} \hat{\E} \left[ (S_T - K)^+ \lvert S_t = S \right] = \exp \{-r (T-t) \} \int^\infty_0 (x - K)^+ \phi (S, t; x, T) dx $\\
$\Rightarrow \frac{\dd C}{\dd K} = -\exp \{-r (T-t) \}\int^\infty_K \phi (S, t; x, T) dx, \frac{\dd^2 C}{\dd K^2} = \exp \{-r (T-t) \} \phi (S, t; x, T) $
\\
With initial condition: $V\lvert_{T = t^*} = (S^* - K)^+,$ in $K>0, T>t^*$\\
\textbf{Dupire with Dividend} $\frac{\dd V}{\dd T} -\frac{1}{2} \sigma^2_{loc} (K, T) K^2 \frac{\dd^2 V}{\dd K^2} + (r-q) \left( V - K \frac{\dd V}{\dd K} \right) - rV = 0$\\

\textbf{Short Rate Model} Assumption: $d r = \mu (r, t) dt + \omega (r, t) dB_t$, then
\begin{align*}
    \frac{\dd Z}{\dd t} + \frac{1}{2} \omega^2 \frac{\dd^2 Z}{\dd r^2} + (\mu - \lambda \omega) \frac{\dd Z}{\dd r} - rZ = 0, Z(r, T; T) = 1 \quad \Leftrightarrow \quad Z(r, t) = \hat{\E} \left[ \left.\exp \left(-\int^T_t r_s ds \right) \right\rvert r_t = r \right]
\end{align*}
\begin{flalign*}
    \textbf{Proof: } &d\left(e^{-\int^t_0 r_s ds} Z(r_t, t; T) \right) = Z(r_t, t) d\left(e^{-\int^t_0 r_s ds}\right) + e^{-\int^t_0 r_s ds} d Z(r_t, t)&\\
    =& -r_t e^{-\int^t_0 r_s ds} Z(r_t, t) dt + e^{-\int^t_0 r_s ds} \left(\frac{\dd Z}{\dd t} dt + \frac{\dd Z}{\dd r} dr_t + \frac{1}{2} \omega^2 \frac{\dd^2 Z}{\dd r^2} dt  \right)&\\
    =& e^{-\int^t_0 r_s ds} \left(\frac{\dd Z}{\dd t} + \frac{1}{2} \omega^2 \frac{\dd^2 Z}{\dd r^2} + (\mu - \lambda \omega) \frac{\dd Z}{\dd r} - rZ + \frac{\dd Z}{\dd r} \omega d\hat{B}_t \right) = e^{-\int^t_0 r_s ds} \frac{\dd Z}{\dd r} \omega d\hat{B}_t&\\
    \Rightarrow& \hat{\E}_t \left[\int^T_t d\left(e^{-\int^t_0 r_s ds} Z(r_t, t; T) \right) \right] = \hat{\E}_t \left[\int^T_t e^{-\int^t_0 r_s ds} \frac{\dd Z}{\dd r} \omega d\hat{B}_t \right] = 0\\
    \Rightarrow& \hat{\E}_t \left[-e^{-\int^t_0 r_s ds} Z(r_t, t; T) + e^{-\int^T_0 r_s ds} Z(r_t, T; T) \right] =0
   \quad \Rightarrow Z(r_t, t; T) = \hat{\E}_t \left[ e^{-\int^T_t r_s ds} Z(r_t, T; T)\right]
\end{flalign*}
\textbf{Tractable Model} Rules of choosing $\mu -\lambda \omega$: 1)analytics solution for ZCB; 2) positive interest rate; 3) mean reversion\\

\textbf{Vasicek} $dr = (\eta - \gamma r) dt + cd\hat{W}_t, Z = e^{A-rB}, B = \frac{1}{\gamma} \left(1 - e^{\gamma (T-t)} \right), A = \frac{1}{\gamma^2} (B-T+t) (\eta \gamma - \frac{1}{2} c^2) - \frac{c^2}{4\gamma} B^2$\\
Satisfies rule 1, 3\\
\textbf{Unique Sol.} $d (e^{\gamma t}r_t) = \gamma e^{\gamma t} r_t dt + e^{\gamma t} d r_t = e^{\gamma t} (\gamma r_t dt + (\eta - \gamma r_t) dt + cd\hat{W}_t) = e^{\gamma t} (\eta dt + cd \hat{W}_t)$\\
$\Rightarrow \int^t_0 d (e^{\gamma s}r_s) = \int^t_0 e^{\gamma s} (\eta ds + cd \hat{W}_s) \Rightarrow r_t = e^{-\gamma t} r_0 + \frac{\eta}{\gamma} (1 - e^{-\gamma t}) + c\int^t_0 e^{-\gamma (t-s)} d\hat{W}_s $\\ 
\textbf{Derivation of $\mathbf{A, B}$} Define $g(t):= r_0 e^{-\gamma t} + \frac{\eta}{\gamma} (1- e^{-\gamma t}), h(t, s):= \sigma e^{- (t-s) \gamma} $, then\\
$Z = \hat{\E} \left[ \left. \exp\left\{-\int^T_t \left( g(s) + \int^s_0 h(s, u) d \hat{W}_u \right) ds \right\}  \right\rvert \mathcal{F}_t \right] = e^{-\int^T_t g(s) ds } \hat{\E} \left[ \left. \exp \left\{-\int^T_t \int^s_0 h(s, u) d \hat{W}_u  ds\right\} \right\rvert \mathcal{F}_t \right] $\\
$=e^{-\int^T_t g(s) ds } \hat{\E} \left[ \left. \exp \left\{-\int^T_0 \int^T_{\max \{u, t \}} h(s, u) ds  d \hat{W}_u  \right\}  \right\rvert\mathcal{F}_t \right]$\\
$= \exp \left\{-\int^T_t g(s) ds -\int^t_0 \int^T_{\max \{u, t \}} h(s, u) ds  d \hat{W}_u \right\}  \hat{\E} \left[ \left. \exp \left\{-\int^T_t \int^T_{\max \{u, t \}} h(s, u) ds  d \hat{W}_u  \right\}  \right\rvert\mathcal{F}_t \right] $\\
$=\exp \left\{-\int^T_t g(s) ds -\int^t_0 \int^T_t h(s, u) ds  d \hat{W}_u \right\}  \hat{\E} \left[ \exp \left\{-\int^T_t \int^T_u h(s, u) ds  d \hat{W}_u  \right\}  \right]$\\
$=\exp \left\{ -\int^T_t g(s) ds - \int^t_0 \int^T_t h(s, u) ds  d \hat{W}_u + \frac{1}{2} \int^T_t \left( \int^T_u h(s, u) ds \right)^2 du \right\}$\\
$=\exp \left\{ -\int^T_t \left( r_0 e^{-\gamma s } + \frac{\eta}{\gamma} (1 - e^{\gamma s}) \right) ds - \frac{\sigma}{\gamma} (1 - e^{-\gamma (T-t)}) \int^t_0 e^{-(t-u) \gamma} d \hat{W}_u + \frac{\sigma^2}{2} \int^T_t e^{2\gamma u} \left( \frac{e^{-\gamma u} - e^{-\gamma T}}{\gamma}  \right)^2  du \right\} $\\
$=\exp \left\{ \frac{r_t}{\gamma} \left(1 - e^{-(T-t)\gamma} \right) + \frac{1}{\gamma} \left(1 - e^{-(T-t)\gamma} \right) \left( r_0 e^{-\gamma t} + \frac{\eta}{\gamma} \left(1 - e^{-(T-t)\gamma} \right) \right) \right\}$\\
$\times \exp \left\{ -\int^T_t r_0 e^{-\gamma s} \frac{\eta}{\gamma} \left(1 - e^{-(T-t)\gamma} \right) ds + \frac{\sigma^2}{2} \int^T_t e^{2\gamma } \left( \frac{e^{-\gamma u} - e^{-\gamma T}}{\gamma}  \right)^2  du  \right\} = \exp \left\{ A(T-t) + r_t B (T-t) \right\}$
\\
\textbf{Ho Lee} $dr = \eta (t) dt + c d \hat{B}_t, Z (r, t; T) = e^{A(t;t) - r(T-t)}, A = -\int^T_t \eta (s) (T-s) ds + \frac{1}{6} c^2 (T-t)^3$\\

\textbf{Hull White} $d r_t = (\eta_t - \gamma r) dt + cd \hat{B}_t$\\

\textbf{General Form} $d r = (\eta (t) - \gamma r) dt + cr^\beta d \hat{B}_t$, e.g. CIR ($\beta = \frac{1}{2}, \eta (t) = \eta$)\\

\textbf{Black, Derman, Toy (BDT)} $d (\log r) = \left(\eta (t) - \frac{\sigma^\prime (t)}{\sigma (t)} \log r \right) + \sigma (t) d \hat{B}_t $\\
\textbf{Black \& Karasinski} $d (\log r) =(\eta (t) - a(t) \log (r) ) dt + \sigma (t) d\hat{B}_t $\\

\textbf{Discrete Coupons} Assume bond with $\mathrm{FV}=1$, coupons $c_i$ at $t_i$, $dr_t = (u - \lambda \omega) dt + \omega d \hat{B}_t$, then\\
With bond price $V = V(r_t, t), t_{i-1} <t <t_i, \frac{\dd V}{\dd t} + \frac{1}{2} \omega^2 \frac{\dd^2 V}{\dd r^2} + (u - \lambda \omega) \frac{\dd V}{\dd r } - rV = 0 $\\
Connection condition: $V(r, t_{i-}) = V(r, t_{i+}) + c_i$; Terminal condition: $V(r, T) = 1 + c_n$\\

\textbf{Continuous Coupons} $\frac{\dd V}{\dd t} + \frac{1}{2} \omega^2 \frac{\dd^2 V}{\dd r^2} + (u - \lambda \omega) \frac{\dd V}{\dd r } - rV +c = 0 , V(r, T) = 1$\\
\textbf{Continuous Dividend} Assume $d S_t =S_t ((r-q) dt + \sigma d\hat{B}_t)$, then $\frac{\dd V}{\dd t} + \frac{1}{2} \sigma^2 S^2 \frac{\dd^2 V}{\dd S^2} + (r-q) \frac{\dd V}{\dd S } - rV = 0$, with terminal condition $V(S, T) = (S-K)^+$\\

\textbf{Discrete Dividend} Assumption: 1) Underlying pays dividend $D$ at $t_d \in (0, T)$, i.e. $S (t_{d^+}) = S (t_{d^-}) - D$; 2) No strike price adjustment, $V(S (t_{d^-}), t_{d^-} ) = V(S (t_{d^+}), t_{d^+} ) = V(S (t_{d^-}) - D, t_{d^+} )$\\
Price of call: $t<T, t\neq t_d, \frac{\dd V}{\dd t} + \frac{1}{2} \sigma^2 S^2 \frac{\dd^2 V}{\dd S^2} + rS \frac{\dd V}{\dd S} - rV = 0$\\
Connection Condition: $V(S, t_{d^-} ) = V(S-D, t_{d^+})$; Terminal Condition: $V(S, T) = (S-K)^+$\\

\textbf{Option on ETF50} Suppose dividend payment not affects total market values of options, rule: 1) Number of options $m = S_{t_{d^-}} / (S_{t_{d^-}} - D) = S_{t_{d^-}} / S_{t_{d^+}}$; 2) New strike price $\Bar{K} = K/m$\\
Then $V(S_{t_{d^-}}, t; K, T) = V (m S_{t_{d^+}} , t; m \Bar{K}, T ) = m V (S_{t_{d^+}} , t; \Bar{K}, T ) $, where $V$ is option value on non-dividend stock\\

\textbf{Convertible Bond} Conversion price decreased by $D$, i.e. new conversion ratio $\Bar{n} = \frac{100}{X-D}$\\
Conversion payoff immediately after dividend payment $\Bar{n} S_{t_{d^+}} = \frac{100}{X - D} (S_{t_{d^-}} - D)$\\
Conversion payoff immediately before dividend payment $n S_{t_{d^-}} = \frac{100}{X} S_{t_{d^-}}$\\
Thus, if $S_{t_{d^-}} >X$, then $\Bar{n} S_{t_{d^+}} > n S_{t_{d^+}}$; if $S_{t_{d^-}} <X$, then $\Bar{n} S_{t_{d^+}} < n S_{t_{d^+}}$\\

\textbf{Cont. Up-out Barrier} $\frac{\dd V}{\dd t}+ \frac{1}{2} \sigma^2 S^2 \frac{\dd^2 V}{\dd S^2} + rS \frac{\dd V}{\dd S} -rV = 0 $\\
Terminal Condition: $V(S, T) = (S-K)^+$; Barrier Condition: $V(H, t) = 0$\\

\textbf{Discrete Up-out Barrier} During $(t_{i-1}, t_i), \frac{\dd V}{\dd t}+ \frac{1}{2} \sigma^2 S^2 \frac{\dd^2 V}{\dd S^2} + rS \frac{\dd V}{\dd S} -rV = 0 $\\
Terminal Condition: $V(S, T) = (S-K)^+ \mathds{1}_{\{S<H\}}$; Connection Condition: $V (S, t_{i^-}) = V (S, t_{i^+})\mathds{1}_{\{S<H\}}$\\

\textbf{Discrete Arithmetic Asian Call} Terminal payoff $\left(\frac{1}{n}\sum^n_{i=1} S_{t_i} -K\right)^+$. 
Define $I_t = \sum_{t_i \leq t} S_{t_i}$, $d I
_t = 0, I_{t_i^+} = I_{t_{i^-}} +S_{t_i}, V(S_{t_{i}}, I_{t_{i^-}}, t_{i^-}) = V(S_{t_{i}}, I_{t_{i^+}}, t_{i^+}) = V(S_{t_{i}}, I_{t_{i^-}} + S_{t_i}, t_{i^+})$\\
$\Rightarrow$ Connection Condition: $V(S, I, t_{i^-}) = V (S, I+S, t_{i^+})$; Terminal Condition $V(S, I, T+) = \left(\frac{I}{n} - K \right)^+$, \\
With PDE model $ \frac{\dd V}{\dd t}+ \frac{1}{2} \sigma^2 S^2 \frac{\dd^2 V}{\dd S^2} + rS \frac{\dd V}{\dd S} -rV = 0$ in $S>0, I \geq 0, t \in (t_{i-1}, t_i)$\\

\textbf{Vol Product} Assume $d S_t = S_t (\mu dt + \sigma dB_t)$ Terminal payoff $f \left[\sum^N_{i=1} \left(\log \frac{S_{t_i}}{S_{t_{i-1}}} \right)^2 \right]$  \\
Define $J_t = \sum_{t_j \leq t} \left(\log \frac{S_{t_j}}{S_{t_{j-1}}} \right)^2 $, then $J_{t_{i^+}} = \sum_{t_j \leq t_{i^+}}  \left(\log \frac{S_{t_j}}{S_{t_{j-1}}} \right)^2 = J_{t_{i^-}} + \left(\log \frac{S_{t_i}}{S_{t_{i -1}}} \right)^2$, for $t \neq t_i, d J_t = 0$\\
Introduce $y_t$ to track previous price, i.e. $y_{t_{i^-}} = S_{t_{i^-}}$, $d y_t = 0$ if $t\neq t_i$ $, y_{t_{t^+}} = S_{t_i}$\\
Thus pricing func. of this product: $V = V(S_t, y_t, J_t, t)$\\
Connection: $V (S_{t_i}, y_{t_{i^-}}, J_{t_{i^-}}, t_{i^-}) = V (S_{t_i}, y_{t_{i^+}}, J_{t_{i^+}}, t_{i^+}) = V(S_{t_i}, S_{t_i}, J_{t_{i^-}} + \left(\log \frac{S_{t_i}}{S_{t_{i -1}}} \right)^2, t_{i^+} ) $\\
Terminal: $V(S, y, J, T^+) = f(J)$ PDE $\frac{\dd V}{\dd t}+ \frac{1}{2} \sigma^2 S^2 \frac{\dd^2 V}{\dd S^2} + rS \frac{\dd V}{\dd S} -rV = 0 $ for $S, y >0, J \geq 0, t\in (t_{i-1}, t_i)$\\
\textbf{Dimension Reduction} Define $x = \frac{S}{y}$, then $V(S, y, J, t) = U (x, J, t)$ \\
Terminal $U(x, J, T^+) = f(J)$ Connection $U (x, J, t_{i^-}) = U (1, J +(\log x)^2 , t_{i^+})$\\
$\frac{\dd V}{\dd t} = \frac{\dd U}{\dd t}, \frac{\dd V}{\dd S} = \frac{\dd U}{\dd x} \frac{\dd x}{\dd S} = \frac{1}{y} \frac{\dd U}{\dd x}, \frac{\dd^2 V}{\dd S^2} = \frac{\dd}{\dd S} \left(\frac{1}{y} \frac{\dd U}{\dd x}  \right) = \frac{1}{y} \frac{\dd^2 U}{\dd x^2} \frac{\dd x}{\dd S} = \frac{1}{y^2} \frac{\dd^2 U}{\dd x^2}  $\\
$\Rightarrow \frac{\dd U}{\dd t} + \frac{1}{2} \sigma^2 x^2 \frac{\dd^2 U}{\dd x^2} + rx \frac{\dd U}{\dd x} - rU = 0$ in $t\in (t_{i-1}, t_i), x>0, J \geq 0$\\

% L7 -----------------------------------------------------------------------------------------------------------------

\textbf{Call Provision} Issuing company has the right to purchase back the bond for a specified amount $M_C$, i.e. $nS \leq V \leq M_C$, which implies $S \leq \frac{M_C}{n}$ and $V \left(\frac{M_C}{n}, t \right) = M_C$\\
Binomial Tree: $V (S, t - \delta t) = \min \left\{ \max \left\{e^{-r \delta t} \left[p V(S_u, t) + (1-p) V (S_d, t) \right], nS  \right\}, M_C \right\}$\\
$\Rightarrow 0 = V(S, t-\delta t) - \min \left\{ \max \left\{ e^{-r\delta t} \left[pV(S_u, t) + (1-p) V(S_d, t) \right], nS \right\}, M_C  \right\}$\\
$\Rightarrow \max \left\{ \min \left\{ V(S, t-\delta t) - e^{-r\delta t} \left[pV(S_u, t) + (1-p) V(S_d, t) \right], V(S, t-\delta t) - nS \right\}, V(S, t-\delta t)-M_C \right\}$\\
Or $\max \left\{ \min \left\{ \frac{V(S, t-\delta t) - e^{-r\delta t} \left[pV(S_u, t) + (1-p) V(S_d, t) \right]}{\delta t}, V(S, t-\delta t) - nS \right\}, V(S, t-\delta t) - M_C \right\}$
\\
Continuous Time Model: $\max \left\{ \min \left\{ -\mathcal{L} V, V(S, t) - nS \right\}V(S, t) - M_C \right\} = 0$, with $V \left(\frac{M_C}{n}, t \right) = M_C, V (S, T)= \max (1, nS)$\\
Equivalently, $-\mathcal{L} V = 0$ if $nS < V < M_C$; $-\mathcal{L} V \geq 0$ if $V = nS$; $-\mathcal{L} V \leq 0$ if $V = M_C$\\

\textbf{Soft Call} Issuer may redeem the CB at a predetermined price $M_C$ if during latest $20$ days, there are at least $10$ days s.t. stock price stays above $1.3X$. $M_C \ll 1.3nX$ implies a forcing conversion upon call.\\
Pricing: An equivalent call boundary $\Bar{H}$:\\ 
$\prob \{\Bar{H}$ is reached: $S_t = S \} = \prob \{S_u \geq H$ satisfying 10 out of 20: $S_t = S \}$\\
Impose a boundary condition at $S = \Bar{H}$: $V (\Bar{H}, t) = \frac{100}{x} \Bar{H}$.\\
\textbf{Put Provision} Holder has the right to return it to the issuer for an amount $M_P$, s.t. $V \geq M_P$\\
\textbf{Vesting Period} E.g. No conversion during first year ($\mathcal{L} V = 0$)\\

\textbf{Dilution} New shares will be issued upon conversion. Suppose $N$ shares and $m$ bonds. Before conversion $NS$(total value)$= NE$(equity)$+ mV$(debt), after conversion, $S = E$.\\
Before conversion, firm's value follows $\frac{d S_t}{S_t} = \mu dt + \sigma d B_t$.
\\Firm's total worth remains unchanged upon conversion: $NS_{t-} = (N+ nm) S_{t+}$, $nm$: new shares issued after conversion; $\frac{N}{N +nm}$: dilution factor\\
If $N \gg nm$, the effect is minimal\\
Constraints: Conversion: $V(S, t) \geq \frac{N}{N +nm} nS$; Bankruptcy permitted: $E \geq 0 \Rightarrow V \leq \frac{N}{m} S$\\
Model: $\max \left\{\min \left\{-\mathcal{L} V, V - \frac{N}{N+nm}nS \right\}, V - \frac{N}{m}S \right\}$ in $S>0, t \in [0, T)$ with $V(S, T) = \min \left\{ \max \left\{1, \frac{N}{N+nm}nS \right\}, \frac{N}{m}S \right\}$\\

\textbf{Pricing with Stochastic Interest Rate} Assume following asset price and interest rate,
\begin{align*}
    dS = S (\mu dt + \sigma dX_1) ; dr = \mu(r, t) dt + w (r, t) dX_2; \E [X_1, X_2] = \rho dt
\end{align*}
Then for $V = V(S, r, t)$, we obtain
\begin{align*}
    dV = \frac{\dd V}{\dd t} dt + \frac{\dd V}{\dd S} dS + \frac{\dd V}{\dd r} dr + \frac{1}{2} \left(\sigma^2 S^2 \frac{\dd^2 V}{\dd S^2} + w^2 \frac{\dd^2 V}{\dd r^2} + 2\rho \sigma S w \frac{\dd^2 V}{\dd S \dd r} \right)
\end{align*}
Consider a portfolio consists of $V$, $-\Delta_1$ of underlying asset, $-\Delta_2$ of ZBC, $\Pi = V - \Delta_1 S - \Delta_2 Z$,
\begin{align*}
    d\Pi = d V - \Delta_1 d S - \Delta_2 d Z = dV - \Delta_1 d S - \Delta_2 \left(\frac{\dd Z}{\dd t} dt + \frac{\dd Z}{\dd r} dr + \frac{1}{2} w^2 \frac{\dd^2 Z}{\dd r^2} dt \right) \leq r\Pi dt
\end{align*}
Eliminate $d S$ and $d Z$ gives $\Delta_2 = \frac{\dd V}{\dd r} / \frac{\dd Z}{\dd r}, \Delta_1 = \frac{\dd V}{\dd S}$, hence
\begin{align*}
    \mathcal{L}V := \frac{\dd V}{\dd t} + \frac{1}{2} \sigma^2 S^2 \frac{\dd^2 S}{\dd S^2} + \rho \sigma S w \frac{\dd^2 V}{\dd S \dd r } + \frac{1}{2} w^2 \frac{\dd^2 V}{\dd r^2} + rS \frac{\dd V}{\dd S} + (\mu - \lambda w) \frac{\dd V}{\dd r} - rV
\end{align*}
with variational inequalities $\min \left\{-\mathcal{L} V, V - nS \right\} = 0, V(S, r, T) = \max \left\{1, nS \right\}$
\\

\textbf{Dual Fund} Suppose has 1:1 structure ($2S = A+B$), share A has $4\%$ coupon rate $\NAV^A_t = 1+4\%t, \NAV^B_t = 2S_t - \NAV^A_t$\\
\textbf{Regular Reset} Coupon payment for Share A, at $t_0 = 1$, e.g. 1 Jan each year\\
1) Holder of each Share A receives $0.04$ (paid in $S$), i.e. $A_{t_0-} = A_{t_0+} + 0.04$; 2) Market price of Share B not change, i.e. $ B_{t_0-} = B_{t_0-}$; 3) To ensure $2S_t = A_t + B_t$, adjust NAV of underlying fund immediately after reset, i.e. $2S_{t_0-} = S_{t_0 +} +0.04 \Rightarrow N_1 = \frac{S_{t_0-}}{S_{t_0-} - 0.02}N$\\
\textbf{Irregular Upside Reset} $S\geq 2$ to keep leverage ratio attractive, $\NAV^B \leq 0.25$ to reduce default risk of Share B holder\\
Suppose one holds $N$ shares S, A, B, resp. Upside reset when $S_{t-} = 2$ at $t=0.5$\\
$S_{t+} = 1$, thus number of share S becomes $2N$.\\
Each share A pays coupon worth $0.02 = 0.04\times \frac{1}{2}$.\\
Each share B pays dividends $1.98$, $\NAV^B_t = 2S_{t} -1 - 4\% t= 2.98 = 1+1.98 $\\
The holder then holds $N$ shares of A, B, resp.\\

\textbf{Irregular Downside Reset} Suppose downside reset when $\NAV^B = 0.25$ at $t=0.5$
$2S_t = \NAV^A_t+\NAV^B_t = 1+4\%t + \NAV^B_t \Rightarrow S_t = 0.635$\\
Thus the number of share S becomes $0.635N$, holder of share A receives $0.02+0.75$, after reset, holder has $\frac{N}{4}$ shares of A, B, resp.\\

\textbf{Formulation} $t=0, \NAV^A_0 = \NAV^B_0 = S_0 = 1$. On normal trading days, $\NAV^A$ grows linearly, $d \NAV^A_t = Rdt$, $(1-\alpha) \NAV^B_t = S_t - \alpha \NAV^A_t$\\
On regular reset dates $\zeta$, coupon paid to share A holder: $\NAV^A_{\zeta-} - 1 = R, \NAV^A_\zeta=1$; $\NAV^B$ remains unchanged; $S_\zeta = S_{\zeta-} - \alpha (\NAV^A_{\zeta-} -1 ) = S_{\zeta-} - \alpha R$\\
\underline{Upside reset} occurs when $S \geq H_u$. On upside reset date $\tau$, early interest payments of A ($\NAV^A_{\tau-}-1$); dividend payments of B ($\NAV^B_{\tau-} -1$); reset of NAVs ($\NAV^A_\tau = \NAV^B_\tau = S_\tau = 1$)\\
\underline{Downside rest} when $\NAV^B \leq H_d$. On downside reset day $\eta$, payments for share A ($\NAV^A_{\eta-} - \NAV^B_{\eta -}$); no payment for share B; after reset ($\NAV^A_\eta = \NAV^B_\eta = S_\eta = 1$); each share A/B become $\NAV^B_{\eta-}$ shares\\

\textbf{Valuation of Share A} 
Assume $d S_t = S_t \left( (r-c)dt + \sigma dW_t \right)$
\begin{align*}
\begin{split}
&V_A (t, S_t) = \E_t \left[ e^{-r(1-t)} (R+ V_A (0, S_1 - \alpha R)) \mathds{1}_{\{1<\tau \wedge \eta \} }\right.\\
&+ \left. e^{-r (\tau-t)} (R\tau + V_A (0, 1)) \mathds{1}_{\{\tau < 1 \wedge \eta\}} \vphantom{e^{-r(1-t)}}+ e^{-r (\eta -t)} (R_{\eta} +1 - H_d + H_d V_A (0, 1)) \mathds{1}_{\{\eta < 1 \wedge \tau\}} \right]
\end{split}
\end{align*}
PDE Model corresponding to $V_A (t, S)$,
\begin{align*}
    \frac{\partial V_A}{\partial t} + \mathcal{L}_{BS} V_A = 0, \quad 0 \leq t <1, \quad H(t)
 < S < H_u,
 \end{align*}
with terminal condition $V_a (1, S) = R + V_A (0, S - \alpha R), H(t) < S < H_u$;\\
boundary condition $V_A (t, H_u) = Rt + V_A (0, 1); V_A (t, H(t)) = Rt + 1- H_d + H_d V_A (0, 1), t\in [0, 1]$
\begin{align*}
    \mathcal{L}_{BS}V_A = \frac{1}{2} \sigma^2 S^2 \frac{\partial^2 V}{\partial S^2} + (r-c) S \frac{\partial V}{\partial S} - rV, \qquad H(t) = \alpha (1+ Rt) + (1-\alpha) H_d
\end{align*}
\textbf{Limitation of Local Vol Model} 1) Local vol models predict market smile moves in opposite direction as price of underlying, which is opposite to typical market behaviour; 2) Sensitive to choice of interpolation scheme used to construct local vol surface; 3) Vol clustering, mean-reverting, negative coll between stock price and vol are not account for real-world dynamics of vol\\
\textbf{Stochastic Vol Model} \underline{Pros}: 1) Replicate vol smile and its evolution over time, leading  to more realistic pricing for exotic options; 2) Hedge adapt to changing vol conditions, leading to more effective risk management\\
\underline{Cons}: Calibration can be computationally demanding; 2) Fewer parameters compared to local vol models
\begin{flalign*}
    &\textbf{Heston Model }d S_t = S_t \left(r dt + \sqrt{v_t} d W_{1, t} \right); d v_t =\kappa (\theta - v_t) dt + \xi \sqrt{v_t} dW_{2, t}; \E [dW_{1, t} dW_{2, t}] = \rho dt, &
\end{flalign*}
$\kappa$: mean reversion speed; $\theta$: long-term variance, $\xi$: volatility of volatility\\
$v(t)$ follows a CIR process; Feller condition $\kappa \theta \geq \frac{\xi^2}{2}$ to make sure $v(t)\geq 0$\\
Proof of $\E [v(t) ] = \theta + (v(0) - \theta  )e^{-\kappa t}$:\\
$d v_t =\kappa (\theta - v_t) dt + \xi \sqrt{v_t} dW_{t} \quad\Rightarrow v (t) = v (0) + \int^t_0 \kappa (\theta - v(s)) ds + \int^t_0 \xi \sqrt{v(s)} dW_{s} $, define $f(t) := \E [v(t)]$\\
$\Rightarrow f(t) = \E \left[v (0) + \int^t_0 \kappa (\theta - v(s)) ds + \int^t_0 \xi \sqrt{v(s)} dW_{s}  \right] = v(0) + \int^t_0 \kappa (\theta - \E [v(s)]) ds = v(0) + \int^t_0 \kappa (\theta - f(s)) ds$\\
$\Rightarrow f^\prime (t) = \kappa (\theta - f(t)) \Rightarrow f^\prime (t) +\kappa f(t) = \kappa \theta \Rightarrow f^\prime (t) e^{\kappa t} +\kappa f(t) e^{\kappa t} = \kappa \theta e^{\kappa t} \Rightarrow \frac{d }{d t} \left( f(t) e^{\kappa t} \right) = \kappa \theta e^{\kappa t} $\\
$f(t) e^{\kappa t} - f(0) = \kappa \theta \int^t_0 e^{\kappa s} ds  \quad\Rightarrow f(t) = v(0) e^{-\kappa t} +\theta - \theta e^{-\kappa t}  $
\begin{flalign*}
    &\textbf{SABR Model } d F_t = \sigma_t F^\beta_t d W_1; d \sigma_t = \alpha \sigma_t d W_2; d W_1 d W_2 = \rho dt&
\end{flalign*}
Often used to model forward rates, can capture volatility smile, density implied by approximation not always arbitrage-free\\
\textbf{Euler Scheme} $S_{t+\Delta t} = S_t + r S_t \Delta t + \sqrt{v_t} S_t \sqrt{\Delta t} Z_1$; $v_{t+\Delta t} = v_t + \kappa (\theta - v_t) \Delta t+ \xi \sqrt{v_t} \sqrt{\Delta t} Z_2$\\

\textbf{Milstein Scheme} Assume $d X_u = a (X_u) du + b (X_u) d W_u $ then by It\^{o}'s lemma,\\
$ d b (X_u) = b^{\prime} (X_u) d X_u + \frac{1}{2} b^{\prime \prime} (X_u) (d X_u)^2 = \left(b^{\prime} (X_u) a(X_u) + \frac{1}{2} b^{\prime \prime} (X_u) b^2 (X_u)  \right) du +b^\prime (X_u) b (X_u) d W_u$\\
Given $b (X_u) -  b(X_t) \approx \int^u_t b^\prime (X_s) b (X_s) d W_s \approx b^\prime (X_t) b (X_t) (W_u - W_t) $, then we obtain,\\
$\int^{t+ h}_t b(X_u) d W_u = \int^{t+ h}_t \left[ b^\prime (X_t) b (X_t) (W_u - W_t) \right] d W_u = b (X_t) (W_u - W_t) + b^\prime (X_t) b (X_t) \int^{t+h}_t (W_u - W_t) d W_u  $,\\
Given $\int^{t+h}_t (W_u - W_t) d W_u = \frac{1}{2} [(W_{t+h} - W_t)^2 -h ] $, then,\\
$\int^{t+ h}_t b(X_u) d W_u = b (X_t) (W_u - W_t) + \frac{1}{2} b^\prime (X_t) b (X_t)  [(W_{t+h} - W_t)^2 -h ]$, where $Z\sqrt{h} = W_{t+h} - W_t$\\
\underline{Example} In Heston's model, $b (v_t) = \xi \sqrt{v_t}, v_{t+ \Delta t} = v_t + \kappa (\theta - v_t) \Delta t + \xi \sqrt{v_t} + \sqrt{ \Delta t} Z_2 + \frac{1}{4} \xi^2 (Z^2_2 -1)\Delta t$
\\
\textbf{Guaranteed Minimum Withdrawal Benefits} \\
\underline{Variable annuity} 1) Issuer: an insurance company (policy issuer); 2) Policyholder initially pays $w_0$ to policy issuer; 3) Policy holder will receive a guaranteed withdrawal; 4) Finite maturity $T$ \\
\underline{Fund-linked} 1) Initial endowment $w_0$ is invested in some fund $S_t$; 2) Investment account $(W_t)$ is created for policyholder\\
\underline{Guaranteed Withdrawal} 1) Policyholder will regularly withdraw a predetermined amount $G$ from $W_t$; 2) If $W_t \leq 0$ for some $t$, the account will be closed but withdrawal will be guaranteed by the policy issuer until $T$\\
\underline{Terminal Payoff} If the account has balance at maturity (after the final withdrawal), then the policyholder will receive the balance\\
Annual management fee is charged by the policy issuer\\
\underline{Assumptions}: 1) Invested fund $S_t$ follows $d S_t = S_t (\mu dt + \sigma dB_t)$; 2) Cont. withdral rate: withdrawal amount during $[t, t+dt]$ is $G dt$; 3) Cont. proportional insurance charge rate $\alpha$: charge during $[t, t+dt]$ is $\alpha W_t dt$; 4) Investment account $W_t$ follows $d W_t = [(\mu - \alpha) W_t -G] dt + \sigma W_t d B_t$\\
\underline{Pricing} Let $V_t$ denote price of GMWB at $t$,
\begin{align*}
    V_t = \hat{\E}_t \left[\int^T_t G e^{-r (s-t)} ds + e^{-r (T-t)} W^+_T \right] = \frac{G}{r} \left(1- r^{-r(T-t)} \right) + \hat{\E}_t \left[e^{-r(T-t)} W^+_T \right]
\end{align*}
Once $V_t$ is known, $\alpha$ may be determined as $V_0 (\alpha) = V (w_0, 0; \alpha) = w_0$\\

Let $U (W_t, t) = \hat{\E}_t \left[e^{-r(T-t) } W^+_t\right]$, $U(W, t)$ solves
\begin{align*}
    \frac{\partial U}{\partial t} + \frac{1}{2} \sigma^2 W^2 \frac{\partial^2 U}{\partial W^2} + [(r-\alpha) W - G] \frac{\partial U}{\partial W} - rU = 0, U(W, T) = W, W>0, t \in [0, T)
\end{align*}
Boundary condition: $\forall t, U(0, t) = 0$; As $W \to \infty, U(W, t) \to W e^{-\alpha (T-t)}$\\
\underline{Discrete Withdrawal} $\alpha$ determined by $U(w_0, 0;\alpha) + \sum^I_{i=1} G_i e^{-r t_i} = w_0$\\
At $t\neq t_i, dW_t = (r-\alpha) W_t dt + \sigma W_t d\hat{B}_t$; At $t=t_i, W_{t_i+} = (W_{t_i -} -G_i)^+$, which implies if $W_u= 0$, then $\forall t>u, W_t \equiv 0$\\
Condition for PDE: 1) if $t=t_i, U(W, t_{i-}) = U ((W-G_i)^+, t_{i+})$; 2) At maturity, $U(W, T-) = (W-G_i)^+$\\

\textbf{Dynamic Withdrawal} Let $A_t$ be the guaranteed balance at $t$. Suppose policyholder withdraws $\gamma_i \leq A_{t_i-}$ at $t_i$. Proportional penalty of $k$ is charged for amount in excess of $G$, i.e. policyholder can receive $f(\gamma_i) = G + (1-k)(\gamma_i-G)$ if $\gamma_i > G$, otherwise $f(\gamma_i) = \gamma_i$. Balance: $A_{t_i+} = A_{t_i-} -\gamma_i$\\

\underline{Pricing} $V_t = \max_{\{\gamma_i\}_{i}} \hat{\E}_t \left[e^{-r (T-t)} W_T + \sum_{t_i \geq t} e^{-r (t_i - t)} f(\gamma_i) \right] $. PDE model: $V(W_t, A_t, t)$\\
Evolution of $W_t$: $t\neq t_i: d W_t = (r-\alpha) W_t dt + \sigma W_t d\hat{B}_t, \quad t=t_i: W_{t_i+} = (W_{t_i-} - \gamma_i)^+$\\
Evolution of $A_t$: $t\neq t_i: d A_t = 0; \quad t = t_i: A_{t_i+} = A_{t_i-} - \gamma_i$
\begin{align*}
    &\text{when }t \neq t_i, \frac{\dd V}{\dd t} + \frac{1}{2} \sigma^2 W^2 \frac{\dd^2 V}{\dd W^2} + (r-\alpha) W \frac{\dd V}{\dd W} - rV= 0,\quad  W>0, A \in (0, w_0)\\
    &\text{when }t = t_i, V(W, A, t_{i-}) = \max_{0\leq \gamma\leq A} \left\{V \left((W - \gamma )^+, A - \gamma_i, t_{i+} \right) + f(\gamma_i) \right\}, \quad  W>0, A \in (0, w_0)
\end{align*}
At maturity, $V(W, A, T-) = \max (W, f(A)), W>0$\\
Boundary condition: $V(W, 0, t) = e^{-\alpha (T-t)}W$, $V(0, A, t) = V_0 (A, t)$, $V(W, A, t) \to e^{-\alpha (T-t)}W$ as $W\to \infty$,
where $V_0 (A, t)$ is determined by
\begin{align*}
    \text{if } t\neq t_i, A \in (0, w_0), \frac{\dd V_0}{\dd t} - rV_0; \quad \text{for }i < I, V_0 (A, t_{i-}) = \max_{0\leq \gamma_i \leq A} \left\{V_0 (A - \gamma_i, t_{i+})+ f(\gamma_i) \right\}
\end{align*}
Terminal condition: $V_0 (A, T-) = f(A)$; Boundary condition: $V_0 (0, t) = 0$\\

\textbf{Individual Fixed Rate Mortgage} Assume mortgage rate $r_M$, for scaled amount $\$1$, constant monthly payment $C: 1 = \sum^{12N}_{i=1} c (1+ r_M/12)^{-i}$, remaining balance: $P = \sum^M_{i=1} c (1+ r_M/12)^{-i}$, \\
where $N$ is number of year to maturity, $M$ is number of remaining payments\\
Cont. rate $r_M$: $1 = \int^T_0 e^{-r_M \tau} C d\tau = -\frac{C}{r_M} \int^T_0 d(e^{-r_M \tau}) = -\frac{C}{r_M} \left(e^{-r_M T}-1 \right) \Rightarrow C = \frac{r_M}{1 - \exp \{ -r_M T\}}$\\
Outstanding balance at $t$: $P(t) = C \int^T_t e^{-r_M (\tau - t)} d\tau$\\

\textbf{Prepayment} Assumptions: 1) Interest rate $dr = \kappa (\mu - r) dt + \sigma \sqrt{r} d \hat{B}_t$; 2) Mortgagor optimally prepays out balance\\
Suppose prepayment with transaction cost $\psi (t) = P(t) (1+X)$\\
\underline{PDE model}: $\max \{-\mathcal{L}V - C, V -\psi \} = 0$ in $r >0, t \in[0, T), \mathcal{L}V = \frac{\dd V}{\dd t} + \kappa (\mu - r)\frac{\dd V}{\dd r} + \frac{1}{2} \sigma^2 r \frac{\dd^2 V}{\dd r^2} - rv$\\
Terminal condition: $V(r, T) = 0$; prepayment region: $\{(r, t): V(r, t) = \psi (t) \}$\\

\textbf{Stanton's Model} Assumption: Endogenous payment: Interest rate concern ($V > \psi (t)$), modeled by a Poisson process with constant intensity $\rho$: $\prob(d L_t = 1) = \rho dt, \prob(d L_t = 0) = (1-\rho) dt$\\
Exogenous payment: those not driven by interest rate. $\prob (d J_t = 1) = \lambda dt, \prob (d J_t = 0) = (1-\lambda) dt$\\
Let $V = V(r, t)$ be the value of an individual mortgage loan whose prepayment behaviour is in average sense,
\begin{align*}
    \text{PDE Model: }V_t + \frac{1}{2} \sigma^2 r V_{rr} + \kappa (\mu - r) V_r - rV + C + \lambda [\psi (t) - V] - \rho [V - \psi (t)]^+ = 0, r>0, t\in [0, T) 
\end{align*}
Subject to $V(r, T) = 0, \lim_{r\to \infty} V(r, T) = 0$\\
Let $M(r, t)$ be the value of an MBS, $\psi (t) = P(t)$,
\begin{align*}
    M_t + \frac{1}{2} \sigma^2 rM_{rr} + \kappa (\mu - r) M_r - rM + C + \lambda [P(t) - M]+ \rho [P(t) - M]\mathds{1}_{\{V>\psi\}} = 0
\end{align*}
Subject to $M(r, T) = 0, \lim_{r \to \infty} M(r, t) = 0$\\
\underline{Default Risk} $d H_t = H_t (\mu_H dt + \sigma_H d B_{tH}), V(r_t, H_t, t)\leq nH_t$\\

\textbf{Public Securities Association Model} Assumption: 1) Prepayment behaviour depends only on age of mortgage; 2) MBS consists of a pool of mortgages, all with the same age and maturity and $r_M$. $C = 1 / \int^T_0 e^{-r_M \tau} d\tau$\\

Remaining Mortgages $Q(t)$: $Q(0) = 1, d Q(t) = -a (t) Q(t) dt \Rightarrow Q(t) = \exp \left\{-\int^t_0 a(s) ds \right\}$\\
Outstanding Balance $P(t)$: $P(t) = C Q(t) \int^T_t e^{-r_M (\tau - t)} d \tau$\\

Fair Value: $\hat{\E}_t \left[\int^T_t \exp \left\{-\int^T_\tau r_s ds \right\} \left[CQ (\tau) + a(\tau) P(\tau) \right] d\tau \right]$\\
With PDE:$\mathcal{L} V +CQ + aP = 0$, where $\mathcal{L}$ depends on interest rate model\\

\textbf{Path-Dependent Model} Assumption: 1) Short term interest rate; \\
2) Conditional prepayment rate depends on short term interest rate: $\mathrm{CPR} = a(t) f(r)$, $a(t), f(r)$ capture dependence on age of mortgage, rate level, resp.\\
Percentage of mortgage left $\frac{d Q_t}{Q_t} =  -a(t) f(r) dt$, outstanding balance $P_t = CQ_t \int^T_t e^{-r_M (\tau - t)}dt$
\begin{align*}
    dP_t = CdQ_t\int^T_t e^{-r_M(\tau-t)} d\tau  + \left[ C Q_t r_M \int^T_t e^{-r_M (\tau-t)} d\tau- CQ_t \right] dt = 
     (r_M P_t - CQ_t-P_t a(t) f(r)) dt
\end{align*}
Value of MBS: $V_t =: V(r_t, P_t, Q_t, t) = \hat{\E}_t \left[\int^T_t \exp \left\{\int^\tau_t r_s ds \right\} [CQ_t + a(\tau) f(r_{\tau} ) P_\tau] d\tau \right]$\\
PDE Model:
\begin{align*}
    \frac{\dd V}{\dd t} + \frac{1}{2} \omega^2 \frac{\dd^2 V}{\dd r^2} + (\mu - \lambda \omega) \frac{\dd V}{\dd r} +
    (r_M P -CQ -a(t) f(r)P) \frac{\dd V}{\dd P}&\\
    - a(t) f(t) Q \frac{\dd V}{\dd Q} -rV + (a(t) f(r) P +CQ)& = 0
\end{align*}
Terminal condition: $V(r, P, Q, T) = 0$

\end{multicols*}
\end{document}
