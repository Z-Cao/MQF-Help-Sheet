\documentclass[12pt,landscape, a4paper]{article}
\usepackage{multicol}
\usepackage{calc}
\usepackage{ifthen}
\usepackage[landscape]{geometry}
\usepackage{hyperref}
\usepackage[english]{babel}
\usepackage{amsthm}
\usepackage{amsmath}
\usepackage{amsfonts}
\usepackage{amssymb}
\usepackage{graphicx}
\usepackage{enumitem, kantlipsum}
\usepackage[nopar]{lipsum}
\usepackage[T1]{fontenc}

\geometry{top=0.45cm,left=0.45cm,right=0.45cm,bottom=0.45cm}

% Turn off header and footer
\pagestyle{empty}
 

% Redefine section commands to use less space
\makeatletter
\renewcommand{\section}{\@startsection{section}{1}{0mm}%
                                {-0.5ex plus -.2ex minus -.2ex}%
                                {0.2ex plus .2ex}%x
                                {\small\small\bfseries}}

\renewcommand{\subsection}{\@startsection{subsection}{1}{0mm}%
                                {-0.5ex plus -.2ex minus -.2ex}%
                                {0.2ex plus .2ex}%x
                                {\scriptsize\scriptsize\bfseries}}
                                
% Don't print section numbers
\setcounter{secnumdepth}{0}
\setlength{\parindent}{0pt}
\setlength{\parskip}{0pt plus 0.5ex}

\theoremstyle{remark}
\newtheorem*{thm}{Thm}
\newtheorem*{defn}{Def}
\newtheorem*{lemma}{Lemma}
\newtheorem*{corollary}{Corollary}
\newtheorem*{question}{Question}
\newenvironment{soln}{\begin{proof}[Solution]}{\end{proof}}

\newcommand{\var}{\operatorname{Var}}
\newcommand{\E}{\operatorname{\mathbb{E}}}
\newcommand{\prob}{\operatorname{\mathbb{P}}}
\newcommand{\cov}{\operatorname{Cov}}

\begin{document}

\setlength{\abovedisplayskip}{0pt}%
\setlength{\belowdisplayskip}{0pt}%
\setlength{\abovedisplayshortskip}{0pt}%
\setlength{\belowdisplayshortskip}{0pt}%
\setlength{\jot}{0pt}% Inter-equation spacing

\fontfamily{phv}

\raggedright
\tiny
%\scriptsize
\begin{multicols*}{3}
\textbf{Cash flow identity}: c.f. from assets = c.f. to creditors + c.f. to stockholders
\\\textbf{Cash flow from asset} = operating c.f. - net capital spending - change in net working capital (NWC), where operating c.f. = earnings before interest and taxes (EBIT) + depreciation - taxes; net capital spending = ending net fixed assets - beginning net fixed assets + depreciation; change in NWC = Ending NWC - beginning NWC
\\\textbf{C.f. to creditors} = interest paid - net new borrowing
\\\textbf{C.f. to stokeholders} = dividends paid - net new equity raised\\
\textbf{Short-term solvency, or liquidity ratios}
\begin{align*}
    &\text{current ratio} = \frac{\text{current asset}}{\text{current liabilities}}; \quad \text{quick ratio} = \frac{\text{current assets - inventory}}{\text{current liability}};\\
    &\text{cash ratio} = \frac{\text{cash}}{\text{current liability}}
\end{align*}
\textbf{Long-term solvency, or financial leverage ratios}
\begin{align*}
    &\text{Total debt ratio} = \frac{\text{total asset} - \text{total equity} }{\text{total asset}}; \quad\text{debt-equity ratio} = \frac{\text{total debt}}{\text{total equity}}\\
    &\text{equity multiplier} = \frac{\text{total asset}}{\text{total equity}}; \quad \text{time interest earned ratio} = \frac{\text{EBIT}}{\text{interest}}\\
    &\text{cash coverage ratio} = \frac{\text{EBIT} + \text{depreciation}}{\text{interest}}
\end{align*}
\textbf{Asset utilisation, or turnover ratios}
\begin{align*}
    &\text{inventory turnover} = \frac{\text{cost of goods sold}}{\text{inventory}}; \quad \text{day's sales in inventory} = \frac{\text{365 days}}{\text{inventory turnover}}\\
    &\text{receivables turnover}  = \frac{\text{sales}}{\text{accounts receivable}}; \quad \text{payable turnover} = \frac{\text{cost of goods sold}}{\text{accounts payable}}\\
    &\text{day's sales in receivable} = \frac{\text{365 days}}{\text{receivables turnover}}; \quad \text{capital intensity} = \frac{\text{total asset}}{\text{sales}}\\
    &\text{day's costs in payable} = \frac{\text{365 days}}{\text{payable turnover}}; \quad \text{total asset turnover} = \frac{\text{sales}}{\text{total asset}}
\end{align*}
\textbf{Profitability ratios}
\begin{align*}
    &\text{profit margin} = \frac{\text{net income}}{\text{sales}}; \text{return on asset(ROA)} = \frac{\text{net income}}{\text{total asset}};\text{ROE} = \frac{\text{net income}}{\text{total equity}}\\
    &\text{Gross Margin} = \frac{\text{Gross Profit}}{\text{Sales}} = \frac{\text{Sales} - \text{COGS}}{\text{Sales}} \quad \text{EBIT or Operating Margin} = \frac{\text{EBIT}}{\text{Sales}}
    \\&\textbf{Dupont Identity: } \text{ROE} = \frac{\text{net income}}{\text{sales}} \times \frac{\text{sales}}{\text{asset}} \times \frac{\text{asset}}{\text{equity}}\\
    & \text{ROE} = \text{Profit margin} \times\text{Total Asset Turnover} \times \text{Equity Multiplier}
\end{align*}
\textbf{Market value ratios}
\begin{align*}
    &\text{price-earning ratio} = \frac{\text{price per share}}{\text{earning per share}}; \quad \text{PS ratio} = \frac{\text{price per share}}{\text{sales per share}}\\
    &\text{market-to-book ratio} = \frac{\text{market value per share}}{\text{book value per share}}; \quad\text{EBITDA ratio} = \frac{\text{enterprise value}}{\text{EBITDA}}
\end{align*}
\textbf{EBITDA}: Earning before interest, taxes, depreciation and amortisation\\
\textbf{Operating Cash Flow} Amount of cash generated by a company's normal business operating activities (sales, costs of products, tax)\\
\textbf{Net Capital Spending} Net amount of buying and selling the \underline{fixed assets} during a period, which indicates the company's investment in the fixed assets\\
\textbf{Change in Net Working Capital} Net change in operating asset and operating liabilities across a specified period, which indicates the company's investment in the current assets and utility of current liabilities\\
\textbf{Profit Margin} Measures firm's operating efficiency. How well does it control costs\\
\textbf{Total Asset Turnover} Measure the firm's asset use efficiency. How well does it manage its assets\\
\textbf{Equity Multiplier} Measure the firm's financial leverage. $\text{EM} = \text{TA} / \text{TE} = 1+ \text{D/E ratio}$

\textbf{Perpetuity}: PV = periodic payment (PMT)/r; Growing perpetuity: $\text{PV} = \text{PMT}/(r-g) $\\
\textbf{(Ordinary) Annuity}: $\text{PV} = \text{PMT} \left[\frac{1 - \frac{1}{(1+r)^2}}{r} \right], \text{FV} = \text{PMT} \left[\frac{(1+r)^t - 1}{r} \right]$
\\Present values of growing annuities: $\text{PV} = \frac{\text{CF}_1}{r-g} \left[1 - \frac{(1+g)^T}{(1+r)^T} \right]$, where $r$ is discount rate, $g$ is growth rate, $\text{CF}_1$ is cash flow in year 1, $T$ is number of period.\\
$\textbf{Effective Annual Rate (EAR)} = \left[1+ \frac{\text{APR}}{m} \right]^m - 1$, where APR is the quoted rate, m is number of compounds per year.\\

\textbf{Annuity} finite series of equal payments occur at regular intervals. Ordinary annuity: first payment occurs the end of the period. 
\begin{align*}
    PV = PMT \left[\frac{1 - \frac{1}{(1+r)^t}}{r} \right] \quad FV = PMT \left[\frac{(1+r)^t - 1}{r} \right]
\end{align*}
Present Value of Growing Annuities $PV = \frac{CF_1}{r-g} \left[1 - \frac{(1+g)^T}{(1+r)^T} \right]$\\
\textbf{Perpetuity} Infinite series of equal payments $PV = PMT / r$\\
Growing Perpetuity $PV = PMT / (r-g)$\\
Dividend Yield $D_{t+1} / P_t$ Capital Gains yield $(P_{t+1} - P_t) / P_t$ \% return $DY+CGY$\\

\textbf{Geometric Average Return} $\left[\prod^T_{i=1} (1+R_i ) \right]^{1/T} -1$,
Under 20 years: arithmetic mean (too optimistic for long run); over 40 years: GAR (too pessimistic for short run)\\

\textbf{Efficient Capital Markets} 1) Stock prices are in equilibrium; 2) Stocks are `fairly' prices; 3) Informational efficiency. No abnormal returns\\
Strong: public or private info; Semistrong: public info; Weak: past prices and volume\\

\textbf{Systemic Risk Principle} The expected required return on an asset depends only on that asset's systemic or market return\\

\textbf{Dividend Growth Model} $p_0 = \frac{D_1}{r_E - g}, r_E >g$ Limitation: a) $r_E < g \Rightarrow$ negative stock price, $r_E = g \Rightarrow$ infinite stock price, neither makes economic sense; b) require $r_E >g$ and $g$ is constant forever.\\

\textbf{Reward-to-Risk Ratio} $\frac{\E [R_j] - R_f}{\beta_j}$\\
\textbf{Security Market Line} Representation of market equilibrium. Expected return obtained by estimating dividends and expected capital gains, implied by asset's expected future cash flows and current price. Require returns obtained from CAPM\\
In Equilibrium, stock price are stable and only change with relevant new info; \textbf{expected return}($r_E = \frac{D_1}{P_0} + g$) must equal \textbf{require return}($r_E = r_f + (r_M - r_f)\beta_E$)\\
\textbf{Market Equilibrium} All assets and portfolio have the same reward-to-risk ratio, each ratio must equal the reward-to-risk ratio for the market $\frac{\E [R_A] -R_f}{\beta_A} = \frac{\E [R_M - R_f]}{\beta_M}$\\

\textbf{Factors Affecting Required Return} $\E [R_j] = R_f + (\E [R_M] -R_f) \beta_j$, $R_f$ measure the pure time value of money; $RP_M = \E [R_M] -R_f$ measure the reward for bearing systematic risk; $\beta_j$ measure the amonut of systematic risk\\

\textbf{Corporate Value Model} (Free Cash Flow Method) Value of firm equals the present value of firm's free cash flows. Step: 1) Finding PV of firm's futurn CFFA, which is firm's market value (MV); 2) MV of common stock = MV - MV of debt - MV of preferred stock; 3) Stock price per share = MV / number of stock\\
Model is preferred as existence of firms don't pay dividend, and dividend are hard to forecast; assume at some point free cash flow will grow at const. rate; \textbf{Terminal value} represents value of firm at point that growth become const.\\

\textbf{Yield to Maturity} $P^B_0 = C \left[\frac{1- 1/(2+YTM)^T}{YTM}\right] + \frac{F_T}{(1+YTM^T)}$ YTM increases, bond prices decrease and vice versa\\
If initial bond price (P)=Par Value (F), then YTM=Coupon rate

\textbf{Current Yield} annual interest paid by a bond expressed as a percentage of its current market price.\\
The weakness of current yield as a bond description is that it does not take into account any capital gain or loss associated with the principal to be paid at maturity\\

If bond sells at: \textbf{Discount} Coupon Rate < Current Yield < YTM; \textbf{Premium} :Coupon Rate > Current Yield > YTM; \textbf{Par Value} Coupon Rate = Current Yield = YTM\\
All else equal, the lower the coupon rate on a bond, the greater is its price sensitivity to changes in interest rates\\

\textbf{Cost of Capital} The cost to a firm for capital funding = the return to the providers of those funds\\
\textbf{Cost of Equity} $r_E$ return required by equity investors given the risk of the cash flows from the firm\\
\textbf{Cost of Debt} $r_D$ cost of debt = the required return on a 
company’s debt
Method 1: Compute YTM on existing debt; method 2: Use estimates of current rates based on the bond rating expected on new debt\\
\textbf{After-tax Cost of Debt} YTM(1-tax rate);
Debt has tax-shield : the interest paid on debt does not need to pay tax $\Rightarrow$ Cost is reduced\\
\textbf{Cost of Preferred Stock} $r_p = D / P_0$\\
\textbf{Weighted Average Cost of Capital} $V = E+P+D$, $\mathrm{WACC} = \frac{E}{V}r_E + \frac{P}{V}r_P + \frac{D}{V} r_D (1 - T_C)$; $T_c$: firm's 
corporate tax rate; $\frac{E}{V}$: \% of common equity in capital structure; $\frac{P}{V}$: \% of preferred stock in capital structure; $\frac{D}{V}$: \% of debt in capital structure\\
Always use the target weights, if possible; Use market value if unavailable\\
Factors that Influence a Company’s WACC: 1) Market conditions, especially interest rates, tax rates and the market risk premium; 2) Firm’s capital structure and dividend policy; 3) Firm's investment policy: Firms with riskier projects generally have a higher WACC\\
A firm’s WACC reflects the risk of an average project undertaken by the firm. Different divisions/projects may have different risks. The division’s or project’s WACC should be adjusted to reflect the appropriate risk and capital structure\\

\textbf{Levered and Unlevered Beta} Unlevered beta=levered beta/[1+D(1-tax)/E]\\
Unlevered beta (a.k.a. Asset Beta or project beta) is the beta of a company without the impact of debt.\\
Levered beta (or “equity beta”) is a measurement that compares the volatility of returns of a company’s relative to market.\\
\textbf{Subjective Approach} If the project is riskier than the firm, use a discount rate greater than the WACC\\
If we expect EBIT to be greater than the break-even point, then leverage is beneficial to our stockholders\\
More leverage $\Rightarrow$ ROE and EPS more sensitive to changes in EBIT\\
\textbf{Homemade Leverage} Investing part personal equity funds and part borrowed funds into an unlevered firm create leverage $\Rightarrow$ Borrow money and buy more shares\\
\textbf{Homemade Un-leverage}  Investing part into levered firm’s equity and part into the firm’s debt (lending) $\Rightarrow$ Sell some shares and lend money\\
\textbf{Modigliani and Miller Proposition}  I: Value of the unlevered firm = Value of the levered firm; II: WACC\\
\textbf{Capital Structure Theory} Case I (Perfect Capital Market): assumption: No corporate taxes \& No bankruptcy costs; proposition: 1) The value of the firm is NOT affected by changes in the capital structure because the cash flows of the firm do not change; 2) The WACC of the firm is NOT affected by capital structure. \\
Case II (Imperfect Capital Market): assumption: Corporate taxes \& No bankruptcy costs; proposition: The WACC of the firm is NOT affected by capital structure. When a firm adds debt, it reduces taxes. The reduction in taxes increases the cash flow of the firm.\\
$V_L = V_U + \text{tax rate}\times D$: Under interest tax shield, Total debt $\nearrow \Rightarrow$ value of firm $\nearrow$\\
M\&M prop I with tax: firm's WACC decreases as firm relies more heacily on debt financing. $\mathrm{WACC} = \frac{E}{V}r_E + \frac{D}{V}r_D (1-T_C)$\\
M\&M prop II with tax: firm's cost of equity $e_E$ rises as firm relies more heavily on debt financing $r_E = r_U + (r_U - r_D) \frac{D}{E} (1-T_C)$\\
Case III: assumption: Corporate taxes \& Bankruptcy costs. $D/E \nearrow \Rightarrow$ prob of bankruptcy $\nearrow \Rightarrow$ expected bankruptcy cost $\nearrow$\\
At some point, the additional value of the interest tax shield will be offset by the expected bankruptcy costs, where the value of the firm will start to decrease and the WACC will start to increase as more debt is added\\
\textbf{Optimal Capital Structure} Case I: No optimal capital structure; Case II: Optimal capital structure = 100\% debt, each additional dollar of debt increases the cash flow of the firm; Case III: Optimal capital structure is part debt and part equity, occurs where the benefit from an additional dollar of debt is just offset by the increase in expected bankruptcy costs\\

\textbf{Static Theory of Capital Structure} Firm borrow up to the point where the tax benefit from an extra dollar in debt is exactly equal to the cost that comes from the increased probability of financial distress.\\

\textbf{Pecking Order Theory} Company use internal financing first, then issue debt if necessary. Equity will be as a last resort (selling stock to raise cash can be expensive).\\
Implication: No optimal debt-equity ratio; Profitable firms use less debt; To avoid selling new equity, company will stockpile internally generated cash (financial slack)\\

\textbf{Net Present Value} (NPV) Sum of the present values of all cash flows. Here, all cash flows include the initial cash flow $\mathrm{NPV} = \sum^n_{t=0} \frac{\mathrm{CF}_t}{(1+r)^t}$\\
Initial cost is $\mathrm{CF}_0$ and often is an outflow $\mathrm{NPV} = \sum^n_{t=1} \frac{\mathrm{CF}_t}{(1+r)^t} - \mathrm{CF}_0$\\
NPV > 0: Project is expected to add value to the firm, will increase the wealth of the owners, thus accept the project\\
NPV is a direct measure of how well this project will meet the goal of increasing shareholder wealth.\\

\textbf{Payback Period} How long does it take to recover the initial cost of a 
project; Decision rule: Accept if the payback period is less than some preset limit\\
Advantage: 1) Easy to understand; 2) Adjusts for uncertainty; 3) Biased towards liquidity\\
Disadvantage: 1) Ignores time value of money; 2) Requires an arbitrary cutoff point; 3) Ignores cash flows beyond the cutoff date; 4) Biased against long-term projects, such as research and development, and new projects\\

\textbf{Average Accounting Return} $\mathrm{AAR} = \frac{\text{Average Net Income}}{\text{Average Book Value}}$; Decision rule: Accept the project if the AAR is greater than target rate.\\
Advantage: 1) Easy to calculate; 2) Needed information usually available\\
Disadvantage: 1) Not a true rate of return; 2) Time value of money is ignored; 3) Uses an arbitrary benchmark cutoff rate; 4) Based on accounting net income and book values, not cash flows and market values\\

\textbf{Internal Rate of Return} (IRR) Discount rate that makes the NPV = 0; Desicion rule: Accept the project if the IRR is greater than the required rate of return\\
Advantage: 1) Preferred by executives, easy to communicate the value of a project; 2) If the IRR is high enough, may not need to estimate a required return; 3) Considers all cash flows and time value of money; 4) Provides indication of risk\\
Disadvantage: 1) Can produce multiple answers; 2) Cannot rank mutually exclusive projects; 3) Reinvestment assumption flawed\\
%\textbf{NPV vs. IRR} NPV and IRR will generally give the same decision; Non-conventional cash flows: Cash flow sign changes more than once (results in multiple IRRs); Mutually exclusive projects: initial investments are substantially different, timing of cash flows is substantially different, will not reliably rank project\\
Conventional cash flow: a series of inward and outward cash flows over time in which there is only one change in the cash flow direction\\
\textbf{Independent vs. Mutually Exclusive Projects} Independent: Cash flows of one project are unaffected by the acceptance of the other; Mutually Exclusive: Acceptance of one project precludes accepting the other\\
\textbf{Reinvestment Rate Assumption} 1) IRR assumes reinvestment at IRR; 2) NPV assumes reinvestment at the firm’s weighted average cost of capital(opportunity cost of capital), more realistic, NPV method is best; 3) NPV should be used to choose between mutually exclusive projects\\

\textbf{Conflicts Between NPV and IRR}: 1) NPV directly measures the increase in value to the firm; 2) Whenever there is a conflict between NPV and another decision rule, always use NPV; 3) IRR is unreliable in the following situations (multiple IRRs): Non-conventional cash flows, Mutually exclusive projects\\

\textbf{Modified IRR} Method: 1) Discounting Approach = Discount future outflows to present (at finance rate) and add to $\mathrm{CF}_0$; 2) Reinvestment Approach = Compound all CFs (at reinvestment rate) except the first one forward to end (FV); 3) Combination Approach = Discount outflows to present, compound inflows to end\\
MIRR will be unique number for each method; MIRR correctly assumes reinvestment at opportunity cost = WACC; MIRR avoids the multiple IRR problem; Managers like rate of return comparisons, and MIRR is better for this than IRR\\

\textbf{Probability Index} $\text{PI} = \sum^T_{t=1} \frac{\text{CF}_t}{(1+ \text{WACC})^t} / \text{CF}_0 = 1 + \text{NPV} / \text{CF}_0$ Measures the benefit per unit cost, based on the time value of money. A profitability index of 1.1 implies that for every \$1 of investment create an additional \$0.10 in value\\
Decision rule: Accept if PI > 1\\
Advantage: 1) Closely related to NPV, generally leading to identical decisions, consider all CFs and time value of money; 2) Easy to understand and communicate; 3) Useful in capital rationing\\
Disadvantage: May lead to incorrect decisions in comparisons of mutually exclusive investments (can conflict with NPV)\\

\textbf{Tax Shield Approach to OCF} (Sales - Costs)(1 - Tax rate) + (Depreciation $\times$ Tax rate)\\
Useful when the major incremental cash flows are the purchase of equipment and the associated depreciation tax shield\\

\textbf{Computing Depreciation} Straight-line depreciation: D = (Initial cost - Residual book value) / \# of year\\

\textbf{Salvage Value} Market price of an asset at the end of its useful life. Different from the (net) book value, which is an accounting value after considering the accumulated depreciation. NBV = initial cost - accumulated depreciation\\
\text{After-Tax Salvage}: If the salvage value is different from the book value of the asset, then there is a tax effect. SV - Tax rate (SV - NBV)\\

\textbf{Scenario Analysis} Examines several possible situations: worst case, base case or most likely case, best case. Provides a range of possible outcomes\\
Problems with Scenario Analysis: Considers only a few possible outcomes; Assumes perfectly correlated inputs, i.e. all `bad' values occur together and all `good' values occur together; Focuses on stand-alone risk, although subjective adjustments can be made\\

\textbf{Sensitivity Analysis} Shows how changes in an input variable affect NPV or IRR. Each variable is fixed except one, change one variable to see the effect on NPV or IRR. Answers `what if' questions
Strengths: Provides indication of stand-alone risk; Identifies dangerous variables; Gives some breakeven information\\
Weaknesses: Does not reflect diversification; Says nothing about the likelihood of change in a variable; Ignores relationships among variables\\

Disadvantages of Sensitivity and Scenario Analysis: Neither provides a decision rule; Ignores diversification, measures only stand-alone risk, which may not be the most relevant risk in capital budgeting\\

\textbf{Sources (Uses) of Cash} Increase (Decrease) long-term debt; Increase (Decrease) equity; Increase (Decrease) current liabilities; Decrease (Increase) current assets; Decrease (Increase) fixed assets\\

\textbf{Operating Cycle} Inventory period + Accounts receivable period, Inventory period = time inventory sits on the shelf (Average inventory / COGS x 365), Accounts receivable period = time it takes to collect on receivables \\
Time required to receive inventory, sell it, and collect on the receivables generated from the sale of the inventory\\
Average Collection Period = Average accounts receivable / Credit sales $\times$ 365\\

\textbf{Cash Cycle} operating cycle - accounts payable period; Accounts payable period = time between receipt of inventory and payment for it (Average accounts payable / COGS $\times$ 365)\\
The time between payment for inventory and receipt from the sale of inventory, measures how long we need to finance inventory and receivables\\

\textbf{Short-Term Financial Policy} Flexible (Restrictive) Policy: Large (Low) amounts of cash and marketable securities; Large (Low) amounts of inventory; Liberal (Little or no) credit policies (large (Low) accounts receivable); Relatively low (high) levels of short-term liabilities $\Rightarrow$ High (Low) liquidity

\textbf{Flexible Financial Policy} Advantage: No difficulty meeting short-term obligations; Cash available for emergencies; Lower shortage costs. Disadvantage: Liquid securities = lower return; Financing S/T assets with L/T debt risky\\
\textbf{Restrictive Financial Policy} Advantage: Higher returns on long term assets; Lower carrying costs; S/T liabilities can be decreased more easily in case of economic downturn. Disadvantage: Less liquidity for emergencies; Higher shortage costs\\
Choosing the Best Policy: Consider cash reserves, maturity dedging; relative interest rates\\

\textbf{Compromise policy} Borrow short-term to meet peak needs, and maintain a cash reserve for emergencies\\
With a compromise policy, the firm keeps a reserve of liquidity that it uses to initially finance seasonal variations in current asset needs. Short-term borrowing is used when the reserve is exhausted.\\

\textbf{Cash Budget} Primary tool in short-run financial planning; Identify short-term needs and opportunities; Identify when short-term financing may be required\\
How it works: Identify sales, cash collections and various cash outflow. Subtract outflows from inflows and determine investing and financing needs\\

\textbf{Line of Credit} The borrower receives a set credit limit; Continuous and repeated process: Borrow and make regular payment; Credit lines tend to have higher interest rates, lower dollar amounts\\

\textbf{Trade Credit}  Interest-free loan: a business is able to buy goods without having to pay till later $\Rightarrow$ firm can postpone the short term financial needs\\

\textbf{Credit Management} trade-off between increased sales and the costs of granting credit\\

\textbf{Component of Credit Policy} Term of sale (credit period, cash discount and discount period, type of credit instrument); Credit analysis, Collection policy\\
Term of Sale 2/10 net 45:  2\% discount if paid in 10 days; Total amount due in 45 days if discount is not taken; Assume 365 days in a year\\

\textbf{Five Cs of Credit} Character: willingness to meet financial obligations; Capacity: ability to meet financial obligations out of operating cash flows; Capital: financial reserves; Collateral: assets pledged as security; Conditions: general economic conditions related to customer’s business\\

\textbf{EOQ Model} (economic order quantity) $\text{EOQ} = \sqrt{\frac{2TF}{CC}}$, $T$ firm's total unit sales per year, $F$ Fixed cost per order, $CC$ Inventory carrying cost per unit\\

\textbf{Cash Dividend} The decision to pay a dividend rests in the hands of the board of directors of the corporation. When a dividend has been declared, it becomes a liability of the firm and cannot be rescinded easily.\\

\textbf{The Irrelevance of Dividend Policy} From investor’s perspective: investors do not need dividends to convert shares to cash; they will not pay higher prices for firms with higher dividends\\

Factors favouring a \textbf{low dividend payout}: 1) In the U.S., \underline{double taxation} is practiced. Earnings are taxed first and then taxed again in the hands of shareholders when they are distributed as dividends. 2) Company might face \underline{restrictions} on its ability to pay dividends. For example, bond holders could forbid dividend payments if dividend is above some level\\

Factors favouring a \textbf{high dividend payout}: 1) It has been argued that many individuals \underline{desire current income} (e.g. retired people, widows and orphans) This group is willing to pay a premium to get a higher dividend yield. 2) Institutions such as university endowment funds and trust funds are \underline{tax exempt}. Such institutions might therefore prefer to hold high-dividend yield stocks. 3) High dividends reduce free cash flow\\
\textbf{Pros} 1) Cash dividends underscore good results and provide support to stock price; 2) Dividends may attract institutional investors who prefer dividend as part of returns; 3) Stock price usually increases with a new or increased dividend; 4) Dividends absorb excess cash and may reduce agency costs\\
\textbf{Cons} 1) Dividends are taxed to recipients; 2) Dividends can reduce internal sources of funding; May force firm to forgo positive NPV projects; May require external financing; 3) Once established, dividends cuts are hard to make without adversely affecting a firm’s stock price\\

\textbf{Stock Repurchase: An Alternative to Cash Dividends} Instead of declaring cash dividends, firms can rid themselves of excess cash through buying shares of their own stock\\
Recently, share repurchase has become an important way of distributing earnings to shareholders\\
Advantage: Flexibility for shareholders; Keeps stock price higher, good for insiders who hold stock options; As an investment of the firm (undervaluation); Tax benefits 

\textbf{Tax Effects} of Dividends and Stock Repurchase: Cash dividends: 1) No investor control over timing or size. 2) Taxed as ordinary income. Repurchase: 1) Allows investors to decide if they want a current cash flow. 2) Taxed only if they choose to sell and reap a capital gain on the sale. 3) Gain may qualify as lower taxed capital gains if shares owned more than one year\\

\textbf{Observations} 1) Aggregate dividend and stock repurchases are massive and have increased steadily; 2) Managers are reluctant to cut dividends; Changes in the dividend signal management’s view concerning the firm’s future prospects; 3) Managers smooth dividends, raising them slowly as earnings grow; 4) Stock prices react to unanticipated changes in dividend; 5) Stock repurchases signal that management believes the current stock price is low\\

\textbf{Stock Dividends} Distribute additional shares of stock instead of cash, increases the number of outstanding shares\\
\textbf{Pros} 1) Company's cash balance remains the same; 2) Decrease in share price may attract new investors; 3) Investors do not owe tax on these dividends until the stock is sold\\
\textbf{Cons} 1) Bonus shares dilute the share price; 2) Stock dividends may signal the company's financial instability; 3) Share dividends may be less attractive to some investors than cash dividends\\

\textbf{Stock Split} Common reason: return price to a `more desirable trading range'\\
\textbf{Reverse Stock Splits} Reason: 1) Transactions costs may be less for investor; 2) Liquidity might be improved; 3) Too low a price not considered `respectable'; 4) Exchange minimum price per share requirement\\

\textbf{M\&A} Process of combining two or more firms into one single entity or acquiring one firm by another\\
Merge: Combination of two firms. An entirely new firm is created\\
Acquisition: One company purchases/takes over another company\\

\textbf{Horizon Expansion} Achieved in its simplest from by adding new sales, warehouse and eventually manufacturing facilities in new geographical areas. Two companies that are in direct competition and share the same product lines and markets\\
Reasons: 1) Increase market share and reduce competition in the industry; 2) Further utilise economies of scale (thus reducing costs); 3) Reshape company’s competitive scope by reducing intense rivalry; 4) Share complementary skills and resources\\

\textbf{Vertical Integration} Achieved by expanding backwards into components, raw materials and services previously obtained from suppliers or forward into products and services produced by customers. A customer and company or a supplier and company.\\
Reasons: 1) Reduce operating costs; 2) Realise higher profits; 3) Ensure tighter quality control; 4) Better flow and control of information along the supply chain; 5) Synergies: operating synergy, financial synergy, managerial synergy, etc.\\

\textbf{Diversification/Conglomeration} Achieved by expanding into unrelated products, services and 
business\\
Reasons: 1) Many widely diversified companies have low market value or low price-to-earning ratio; 2) Investors investment risk is reduced, but it might not be better than the portfolio diversification; 3) Management team might not be efficient to govern two types of business from different industries\\

\textbf{Hostile Takeover} Occurs when an acquiring company attempts to take over a target company against the wishes of the target company's management. May take place if a company believes a target is undervalued or when active shareholders want changes in a company\\
In a hostile takeover, the acquirer goes directly to the company's shareholders or fights to replace management to get the acquisition approved\\
\textbf{Tender Offer} a bid to purchase some or all of shareholders' stock in a corporation.\\
Investor normally offers a higher price per share than the company’s stock price, providing shareholders a greater incentive to sell their shares. \\
If a privately or publicly traded company executes a tender offer directly to shareholders without the board of directors‘ consent, then it becomes a `hostile' takeover\\

\textbf{Proxy fight} Persuade shareholders to vote\\
A proxy fight refers to the act of a group of shareholders joining forces and attempting to gather enough shareholder proxy votes to win a corporate vote\\
The acquiring company employs a third-party proxy solicitor, the proxy solicitor may reach out to each stakeholder individually and state the acquirer's case and influence shareholder votes\\

\textbf{Defences} 1) Poison pill: shareholder right plan. With a poison pill strategy, existing shareholders (not the hostile acquirer)can purchase additional shares at steeply discounted prices. 2) Golden parachutes: Lucrative compensation packages inked into the contracts of top executives that compensate them when they are terminated from M\&A.\\

\textbf{Merger Control: Anti-trust Law} Regulations that encourage competition by limiting the market power of any particular firm. This law often involves ensuring that mergers and acquisitions don’t overly concentrate market power or form monopolies.\\

\textbf{Taxable acquisition} Shareholders in a the target firm are considered selling their shares: capital gain will be taxed. $\Rightarrow$ cash acquisitions are generally taxable\\
Capital gain effect: Target firm shareholders might require higher price to compensate the tax payout\\
If $\Delta V = V_{AB} - (V_A + V_B)$, the acquisition is to generate synergy\\
Factors to have $\Delta V$: Revenue enhancement, Cost reduction, Lower tax, Reduction in capital needs\\
\textbf{Lower taxes} 1) Unused Debt capacity: Some firms do not use debt even if they are able. $\Rightarrow$ Adding debt after M\&A provide tax benefits. 2) Net Operating Losses: Firms that lose money on a pretax basis need not pay taxes. $\Rightarrow$ end up with tax losses that the firm cannot use. $\Rightarrow$ Merged with a firm with positive tax liabilities can be more valuable.\\
\underline{Tax authority may disallow an acquisition if the principal purpose of the acquisition is to avoid tax}\\ 

\textbf{Reduction in Capital Needs} 1) Reduce the combined investments needed: Firm A may buy Firm B if it is cheaper than to build from scratch. 2) Reduce working capital from more efficient management in cash, accounts receivable and inventory.\\

Write-up effect: in US, the target firm could be re-valued (or written up) from historic book value to the estimated current market value. Thus, tax effect could be incurred.\\

\textbf{Tax-free acquisition} Transaction is considered as an exchange instead of a sale, no capital gain or loss occurs thus no tax. $\Rightarrow$ stock acquisitions are generally tax free




























\end{multicols*}
\end{document}
