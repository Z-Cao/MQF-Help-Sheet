
\documentclass[10pt,landscape, a4paper]{article}
\usepackage{multicol}
\usepackage{calc}
\usepackage{ifthen}
\usepackage[landscape]{geometry}
\usepackage{hyperref}
\usepackage[english]{babel}
\usepackage{amsthm}
\usepackage{amsmath}
\usepackage{amsfonts}
\usepackage{amssymb}
\usepackage{graphicx}
\usepackage{enumitem, kantlipsum}
\usepackage[nopar]{lipsum}
\usepackage{upgreek}
\usepackage{mathtools}
\usepackage{dsfont}
% \usepackage{unicode-math}

\geometry{top=0.4cm,left=0.4cm,right=0.4cm,bottom=0.4cm}

% Turn off header and footer
\pagestyle{empty}
 

% Redefine section commands to use less space
\makeatletter
\renewcommand{\section}{\@startsection{section}{1}{0mm}%
                                {-0.5ex plus -.2ex minus -.2ex}%
                                {0.2ex plus .2ex}%x
                                {\normalfont\normalsize\bfseries}}

\renewcommand{\subsection}{\@startsection{subsection}{1}{0mm}%
                                {-0.5ex plus -.2ex minus -.2ex}%
                                {0.2ex plus .2ex}%x
                                {\small\small\bfseries}}
                                
% Don't print section numbers
\setcounter{secnumdepth}{0}
\setlength{\parindent}{0pt}
\setlength{\parskip}{0pt plus 0.5ex}

\theoremstyle{remark}
\newtheorem*{thm}{Thm}
\newtheorem*{defn}{Def}
\newtheorem*{lemma}{Lemma}
\newtheorem*{corollary}{Corollary}
\newtheorem*{question}{Question}
\newtheorem*{Ex}{Ex}
\newenvironment{soln}{\begin{proof}[Solution]}{\end{proof}}

\newcommand{\var}{\operatorname{var}}
\newcommand{\E}{\operatorname{\mathbb{E}}}
\newcommand{\prob}{\operatorname{\mathbb{P}}}
\newcommand{\cov}{\operatorname{Cov}}
\newcommand{\abs}[1]{\left\lvert #1 \right\rvert}
\newcommand{\F}{\mathcal{F}}
\newcommand{\Q}{\mathbb{Q}}


\begin{document}

\setlength{\abovedisplayskip}{0pt}%
\setlength{\belowdisplayskip}{0pt}%
\setlength{\abovedisplayshortskip}{0pt}%
\setlength{\belowdisplayshortskip}{0pt}%
\setlength{\jot}{0pt}% Inter-equation spacing

\scriptsize
\raggedright


\begin{multicols*}{3}
\textbf{Portfolio} $h = (\alpha, \beta) \in \mathbb{R}^2$. Value of portfolio at $t$ $V^h_t = \alpha B_t + \beta S_t$
\begin{flalign*}
    & \textbf{Price in Bino Tree} e^{-rT} \sum^T_{k=0} \begin{pmatrix}T\\K\end{pmatrix} q_u^k q_d^{T-k} \phi (su^k d^{T-k}) \quad q_u = \frac{e^{rT}-d}{u-d} &
\end{flalign*}
Some properties of cond. expectation: a)(tower) If $\mathcal{H} \subset\mathcal{G}$, then $\E [\E [Y\lvert \mathcal{G}] \lvert \mathcal{H} ] = \E [Y\lvert \mathcal{H}] = \E [\E [Y\lvert \mathcal{H} ]\lvert \mathcal{G}]$; b) $\E [XY \lvert \mathcal{G}] = \E [X \lvert \mathcal{G} ] = \E [X\lvert \mathcal{G}] Y$ for $\E [\lvert XY \rvert ]<\infty$ and if $Y$ is $\mathcal{G}$-measurable.\\

\textbf{Markov property}: $\prob (X_t \in A \lvert \mathcal{F}_s ) = \prob (X_t \in A \lvert X_s ), \forall A$.
For any bounded and cont func $f$, $\E [f(X_t)\lvert \mathcal{F}_s ] = \E [f(X_t)\lvert X_s ].$\\

A stochastic process $\{X_t\}$ with $\E [\lvert X_t \rvert ] < \infty$, $\forall t \geq$ is called a \textbf{martingale} if $\E [X_t \lvert \mathcal{F}_s] = X_s$ holds for every $0 < s < t < \infty$.\\
\textbf{Submartingale} $\E [X_t \lvert \mathcal{F}_s] \geq X_s$, \textbf{supermartingale} $\E [X_t \lvert \mathcal{F}_s] \leq X_s$\\

If $Z$ is a random variable with $\E [\lvert Z\rvert] < \infty$. Then a \textbf{doob martingale} can be defined as $X_t = \E [Z \lvert \mathcal{F}_t ]$. For every $0 < s < t < \infty$, $\E [X_t \lvert \mathcal{F}_s ] = \E [\E [Z\lvert \mathcal{F}_t ]\lvert \mathcal{F}_s] = \E [Z \lvert \mathcal{F}_s ] = X_s$.\\

A stochastic process $\left(W_t \right)_{t\geq 0}$ is called a \textbf{Wiener process} if a) $W_0 = 0$; b) $W$ has independent increments, i.e. if $r < s \leq t < u$ then $W_u - W_t$ and $W_s - W_r$ are independent; c) For $s < t$, $W_t - W_s \sim \mathcal{N}(0, t-s)$; d) Every path $t \to W_t$ is cont.
$W$ is a martingale; if $s < t, \cov (W_s, W_t) = s$\\
$\forall a>0$, $X_t := W_{a^2t}$ is a Wiener Process\\
Generalised Wiener Process: $X_t = at + b W_t, a \in \mathbb{R}, b > 0$\\

\textbf{Donsker's Thm}: The process $W_t^{(n)} := S_{\lfloor nt \rfloor} / \sqrt{n}$ `converges' to a Wiener process, $S = (S_n)_{n \in \mathbb{N}}$ and $S_n = \sum_{i=1}^n X_i$\\

\textbf{Covariation} (\textbf{Quadratic Variation}) of $f, g$: 
\vspace{-3pt}
$$[f, g](T) := \lim_{\lVert \Pi \rVert \to 0} \sum_{i=1}^k ( f(t_i) - f(t_{i-1})) (g(t_i) - g(t_{i-1}))$$
Partition $\Pi = \{t_0 , \dots , t_k \}, \lVert \Pi \rVert := \max_{i\in [1, k]}\abs{t_i - t_{i-1}}$\\
$\E [Q(W, \Pi)] = \E[\sum^k_{i=1} (W_{t_i} - W_{t_{i-1}})^2 ] = T$\\
$dtdt = 0, dt dW_t = 0, dW_t dW_t = dt$\\
\vspace{-3pt}
If for all $n \in \mathbb{N}$ we have $\E [X_n] = a, \var (X_n) \to 0$, then $X_n \xrightarrow{L_2 } a$\\

\textbf{Simple Process} Stochastic process $X_t$ if it has the form $X_t = \xi_0 1_{t=0} + \sum^{n-1}_{j=0} \xi_j 1_{t_j < t \leq t_{j+1}}$, where $\xi_j \equiv \xi_j (\omega) \in \F_{t_j}$  s.t. $\E [\xi^2_j] < \infty$ \\

\textbf{Stochastic Integral of Simple Process} if $t_m < t \leq t_{m+1}$,
\begin{align*}
    \int^t_0 X_s dW_s &= \sum^{m-1}_{j=0} \xi_j (W_{t_{j+1}} - W_{t_j}) + \xi_m (W_t - W_{t_m})\\
    &= \sum^{n}_{j=0} (W_{t\wedge t_{j+1}} - W_{t\wedge t_j}), \quad \text{where }a\wedge b = \min(a, b) 
\end{align*}
$\int^t_0 X_s dW_s$ is a \textbf{martingale}, $\E [\int^t_0 X_s dW_s] = 0$\\
\textbf{Ito Isometry} $\var (\int^t_0 X_s dW_s) = \E [(\int^t_0 X_s dW_s)^2] = \E [\int^t_0 X^2_s ds]$\\
\textbf{Quadratic Variation} $I (t) = \int^t_0 X_s dW_s, \quad [I, I]_t = \int^t_0 X^2_s ds$\\

\textbf{Stochastic Integral for $\mathcal{L}^2$ Integrands}
\begin{align*}
    \int^T_0 X_s d W_s := I(t) =\lim_{n \to \infty} I_n (T) =\lim_{n \to \infty } \int^T_{0} X_s^{(n)} d W_s
\end{align*}
\textbf{Computation} Let $X$ be Riemann-integrable, left-continuous and adapted,
\begin{align*}
    \int^T_0 X_s dW_s = \lim_{\lVert \Pi \rVert \downarrow 0} \sum^{n-1}_{i=1} X_{s_i} (W_{S_{i+1}} - W_{S_i}), \Pi = \{0 =s_0 < \cdots < s_n =T \}
\end{align*}
% L5 ------------------------------------------------------------------------------------------
\textbf{Ito's Lemma} $dV = \frac{\partial V}{\partial t} dt + \frac{\partial V}{\partial S} dS_t + \frac{1}{2} \frac{\partial^2 V}{\partial S^2} (dS_t)^2$, $(dS_t)^2 = (\mu dt + \sigma dW_t)^2 = \left(\mu (dt)^2 +2\mu \sigma dt dW_t + \sigma^2 (dW_t)^2\right) = \sigma^2 dt$\\
\textbf{Conditions on the coef} $\mu , \sigma$ are said to satisfy the \textbf{Lipschitz condition} if $\forall t \in [0, \infty), x, y$, 
$\abs{\mu(t, x) - \mu(t, y)} + \abs{\sigma(t, x) - \sigma(t, y)} \leq C \abs{x-y}$;\\
\textbf{Linear growth cond} if $\forall t \in [0, \infty), x, y$, $\abs{\mu (t, x)} + \sigma{\mu (t, x)} \leq D (1 + \abs{x})$.\\

\textbf{Geometrix Brown Motion} $dS_t = S_t (\mu dt + \sigma dW_t), S_0 = s, S_t = S_0 exp\left\{(\mu - \sigma^2 / 2)t + \sigma W_t \right\}$\\

\textbf{Self-financing} Portfolio $h_t = (\alpha_t, \beta_t)$, $d V_t = \alpha_t d B_t + \beta_t d S_t, V_t = \alpha_t B_t + \beta_t S_t$\\

\textbf{Adapted process} $(X_t)_{t\geq 0}$ is \textbf{adapted} to $(\mathcal{F}_t)_{t\geq 0}$ if $X_t \subseteq \mathcal{F}_t \forall t \geq 0$\\

\textbf{Arbitrage portfolio} Self-financing portfolio $h:= (h_t)_{0\leq t \leq T}$ s.t. i) $V^h_0 = 0$, ii) $\prob{V^h_0 \geq 0} = 1$ iii) $V^h_0 \geq 0$ with (strictly) positive probability\\

\textbf{Black-Scholes model} Dynamics of risk-free asset: $d B_t = rB_t dt$; Dynamics of stock: $dS_t = S_t (\mu dt + \sigma dW_t )$\\
\textbf{BS Equation} $\frac{\partial F}{\partial t} + r \frac{\partial F}{\partial S} S + \frac{1}{2} \frac{\partial^2 F}{\partial S^2} \sigma^2 S^2 - rF = 0, F(T, S) = \phi (S)$\\

\textbf{Feynman-Kac} Assume $f(t, x)$ satisfies $\qquad \E_{s, \bar{x}}[\cdot] = \E [\cdot \lvert X_s = \bar{x}]$
\begin{align*}
    \frac{\partial f}{\partial t} (t, x) + \mu (t, x) \frac{\partial f}{\partial x} + \frac{1}{2} \frac{\partial^2 f}{\partial x^2} (t, x) \sigma^2 (t, x) = 0; f(T, x) = \phi(x)
\end{align*}
Then $f(s, \bar{x}) = \E_{s, \bar{x}} [\phi(X_t)]$, where $dX_t = \mu (t, X_t) dt + \sigma (t, X_t) dW_t, X_s = \bar{x}$\\
\textbf{Corollary} Let $g(t, x):= e^{-r(T-t)} f(t, x)$ satisfy
\begin{align*}
    \frac{\partial g}{\partial t}(t,x) + \mu(t,x)\frac{\partial g}{\partial x}(t,x) + \frac{1}{2}\frac{\partial^2 g}{\partial x^2}(t,x) \sigma^2(t,x) - rg(t,x) = 0, g(T, x) = \phi(x)
\end{align*}
Then $f(t, x)$ satisfies the Feynman-Kac equation:
\begin{align*}
    \frac{\partial f}{\partial t}(t,x) + \mu(t,x)\frac{\partial f}{\partial x}(t,x) + \frac{1}{2}\sigma^2(t,x)\frac{\partial^2 f}{\partial x^2} (t,x) = 0, f(T, x) = \phi(x)
\end{align*}
Therefore, $g(t, x) = e^{-r(T-t)}f(t, x) = e^{-r(T-t)}\mathbb{E}_{t,x}[\phi(X_T)]$, where $X_t = x$\\

\textbf{Risk-Neutral Valuation} The price at time $t$ of the derivative with payoff $\phi (S_T)$ is $e^{-r(T-t)} \E_{t, s} [\phi (S_T)]$, where $d S_u = S_u (rdu+ \sigma dW_u ), S_t = s$\\

\textbf{BS Formula} Price of call option at time $t$ is $S_tN(d_1 ) - e^{-r(T-t)} KN(d_2 )$, Price of put option: $Ke^{-r(T-t)} N(-d_2 ) - S_t N(-d_1)$, $d_1 = d_2 + \sigma \sqrt{T-t}, d_2 = \frac{1}{\sigma \sqrt{T-t}} \left[\log \left(\frac{S_t}{K} \right) + (r - \frac{1}{2} \sigma^2 )(T-t) \right]$\\
\textbf{Put-Call Parity} $c + Ke^{-rT} = p + S_0$\\

\textbf{Martingale Representation Thm} Let $W_t$ be a Wiener process and $\F := (\F_t)_{t\geq 0}$ generated by $W_t$. Let $M:= (M_t)_{t\geq 0}$ be a martingale with respect to $\F$. Then $\exists h$ adapted to $\F$ s.t. $M_t=M_0 + \int^t_0 h_s d W_s$.\\

\textbf{Girsanov Thm} Let $W_t^{\prob} $ be a Wiener process under $\prob $. Let $k_t$ be an adapted process and define $dL_t = k_t L_t d W^{\prob}_t, L_0 = 1$\\
Assume $\E^{\prob} [L_T] = 1$ and define $\Q$ on $\F_t$ by $L_T = \frac{d\Q}{d\prob}$. Then $dW^{\Q}_t = dW_t^{\prob} - k_t dt$ is a Wiener process under $\Q$.\\
$$L_t = \exp \left\{\int^t_0 k_s dW^{\prob}_s  - \frac{1}{2}\int^t_0\abs{k_s}^2 ds \right\}$$
\textbf{Novikov Condition} If $\E^{\prob} [\exp (\frac{1}{2} \int^T_0 \abs{k_t}^2 dt)] < \infty $, then $\E^{\prob} [L_T] = 1$\\

\textbf{Multidimensional Ito} $S^0, \cdots, S^n$ be ito process, $V(t, S^0, \cdots, S^n)$ be a smooth function, then $\text{d}V = \frac{\partial V}{\partial t}\text{d}t + \sum_{i=0}^{n} \frac{\partial V}{\partial S^i}\text{d}S^i + \frac{1}{2} \sum_{i,j=0}^{n} \frac{\partial^2 V}{\partial S^i \partial S^j}\text{d}S^i\text{d}S^j.
$\\

\textbf{Risk-neutral measure} In the BS model, $\frac{S_t}{B_t}$ is a martingale under $\Q$.\\

\textbf{Market Completeness} Let $X$ be a payoff at time $T$. IF there exists a self-financing portfolio $h$ that replicates $X$, then the market is said to be complete. BS model is complete.
\begin{align*}
    \Delta = \frac{\partial P}{\partial S} (t, S_t), \Gamma = \frac{\partial^2 P}{\partial S^2} (t, S_t), \Theta = \frac{\partial P}{\partial t} (t, S_t), \nu = \frac{\partial P}{\partial \sigma} (t, S_t)
\end{align*}
In BS model, $\Delta = N(d_1), \Gamma = \frac{f(d_1)}{S_t \sigma \sqrt{T-t}}, f(x) = \frac{1}{\sqrt{2\pi}} e^{-x^2 / 2}$\\

\textbf{BS with dividends} $\frac{\partial F}{\partial t} + (r - q)\frac{\partial F}{\partial S}S + \frac{1}{2} \frac{\partial^2 F}{\partial S^2} \sigma^2 S^2 - rF = 0.
$\\
$Y_t:= \frac{S_t}{B_t} + \int^t_0 \frac{1}{B_u} dD_u$ is a martingale under $\Q$\\

\textbf{Change of Numeraire} Let $N_t$ be numeraire, then $\exists$ measure $\mathbb{Q}^N$ s.t. every Euro derivative with payoff $X$ at time $T$ satisfies $\frac{V_t}{N_t} = \E^{\mathbb{Q}^N}_t \left[ \frac{X}{N_T} \right]$, $V_t$ is price of $X$ at time $t$\\
Any positive \textbf{financial asset} can be chosen as numeraire
\begin{Ex}
    Let the price dynamics of a non-dividend stock under $\Q$ be $dS_t = S_t (r dt + \sigma dW^{\Q}_t)$. What are dynamics of $S_t$ under $\Q^s$ with the stock as numeraire?\\
    \begin{align*}
        &L_T = \frac{d \Q^S}{d\Q} = \frac{B_0}{B_t} \frac{S_T}{S_0} = e^{-rT} e^{(r-\frac{1}{2}\sigma^2)T + \sigma W_T}= e^{\sigma W_T - \frac{1}{2}\sigma^2 T} \\
        &= e^{\int^T_0 \sigma dW_t - \frac{1}{2} \int^T_0 \sigma^2 dt} \quad \text{choose $\sigma = k_t$ and apply Girsanov},\\
        &d W^{\Q^S}_t = dW^{\Q}_t - \sigma dt \quad \text{is a Wiener process under $\Q^S$}\\
        &dS_t = rS_t dt + \sigma S_t d W^{\Q}_t = (r+\sigma^2) S_t dt + \sigma S_t d W^{\Q^S}_t
    \end{align*}
\end{Ex}

\textbf{American option} $C, P -$ price of Amer option, $c, p -$ price of Euro option\\
\textbf{Bound of Call} $\left(S_t - K e^{-r (T-t)} \right)^{+} \leq c(S_t, t) \leq C(S_t, t) \leq S_t$; If interest rates are positive and stock pays no dividends then $c(S_t, t) = C(S_t, t)$\\
\textbf{Bound of Put} $\left(Ke^(-r(T-t)) - S_t \right)^+ \leq s(S_t, t) \leq P(S_t, t) \leq K$\\

\textbf{Fokker-Planck Equation} Let $(X_t)_{t\geq 0}$ be a process with dynamics $d X_t = \mu (t, X_t) dt + \sigma(t, X_t) d W_t$. For $s \leq t$, let $p (s, y; t, x)$ be the pdf of $X_t$, when $X_s = y$. Then $p (s, y; t, x)$ satisfies
\begin{align*}
    \frac{\partial}{\partial t} p = - \frac{\partial}{\partial x} [\mu(t, x) p(s, y; t, x)] + \frac{1}{2} \frac{\partial^2}{\partial x^2} [\sigma^2(t, x) p (s, y; t, x)]
\end{align*}

\begin{flalign*}
&\textbf{Dupire Formula} \quad \sigma^2 (T, K) = 2\frac{\frac{\partial}{\partial T} c(T, K) + K r \frac{\partial}{\partial K} c(T, K) }{K^2 \frac{\partial^2}{\partial K^2} c(T, K)} &
\end{flalign*}

\textbf{Stopping Time} Let $\F := (\F_t )_{t \geq 0}$ be a filtration. A stopping time w.r.t $\F$ is a non-negative random variable $\tau$ s.t. $\{\tau \leq t \} \in \F_t, \forall t \geq 0$\\
Let $\tau \in [t, T]$ be a stopping time, and want to solve $\sup_{\tau} \E [Z_{\tau}]$, 1) if $(Z_t)_{t\geq 0}$ is a submartingale, then $Z_t \leq \E [Z_{\tau} \lvert \F_t]$, maximiser is $\hat{\tau} = T$; 2) if $(Z_t)_{t\geq 0}$ is a supermartingale, then $Z_t \geq \E [Z_{\tau} \lvert \F_t]$, maximiser is $\hat{\tau} = 0$; 3) if $(Z_t)_{t\geq 0}$ is a martingale, then $Z_t = \E [Z_{\tau} \lvert \F_t]$, any stopping time is optimal\\

\textbf{Optimal Stopping} For each $\tau$, define value func $J_t (\tau) = \E [Z_{\tau} \lvert \F_t]$. Define optimal value func $V_t = \sup_{t \leq \tau T} \E [Z_{\tau} \lvert \F_t]$. Stopping time $\hat{\tau}$ with $V_t$ is optimal at $t$\\

The process $X$ \textbf{dominates} process $Y$ if $X_t \geq Y_t$ almost surely for all $t\geq 0$\\

The \textbf{Snell Envelop} of process $Y$ is the smallest supermartingale that dominates $Y$\\
The optimal value process $V$ is the Snell Envelop of $Z$\\

\textbf{Linear Complementarity Problem} Let $\phi (S_t)$ be the payoff of the option. If $dV_t \geq rV_t dt$, there is an arbitrage opportunity. By no arbitrage, $F(t, S_t) \geq \phi(S_t)$. If $F(t, S_t) > \phi(S_t)$, then it is not optimal to exercise the option, and $\frac{\partial F}{\partial t} + r \frac{\partial F}{\partial S} S + \frac{1}{2} \frac{\partial^2 F}{\partial S^2} \sigma^2 S^2 - rF = 0$\\
If $F(t, S_t) \geq \phi(S_t)$, then it is optimal to exercise and we have $dV_t \leq rV_t dt$, or $\frac{\partial F}{\partial t} + r \frac{\partial F}{\partial S} S + \frac{1}{2} \frac{\partial^2 F}{\partial S^2} \sigma^2 S^2 - rF \leq 0$\\
Above are two complementary cases. The price $F(t, S_t)$ of American option,
\begin{align*}
    \min \left\{-\frac{\partial F}{\partial t} - r \frac{\partial F}{\partial S} S - \frac{1}{2} \frac{\partial^2 F}{\partial S^2} \sigma^2 S^2 + rF, F - \phi \right\} = 0, F(T, S) = \phi(S),
\end{align*}
where the domain is $D = [0, T) \times (0, \infty)$\\

\textbf{Free Boundary Value Problem} Consider the modified optimal value function $V(t, s) = \sup_{t \leq \tau \leq T} \E^{\Q}_{t, s} [e^{-r(\tau - t)} (K - S_{\tau})^+]$, $V$ is the price at $t$ of American put option. Write the problem as a free boundary value problem,
$\frac{\partial}{\partial t}V+ rs \frac{\partial}{\partial s}V + \frac{1}{2} \sigma^2 s^2 \frac{\partial^2}{\partial s^2}V -rV=0, (t, s)\in C, V(t, s) = (K-s)^+ \in \partial C$\\
$C = \{(t, x): V(t, x) > \phi(t, x) \}$ is the \textbf{continuation region}\\
The problem is formulated as, 
\begin{flalign*}
    & \text{if } S>S^P_* (t), \text{then } \frac{\partial P}{\partial t} + rS \frac{\partial P}{\partial S} + \frac{1}{2} \sigma^2 S^2 \frac{\partial^2 P}{\partial S^2} -rP = 0  &\\
    & P(t, S_*^P (t)) = K - S_*^P (t) = (K - S_*^P (t))^+ \quad P(T, S) = (K-S)^+ &
\end{flalign*}
Finding the exercise boundary is part of the problem.\\

\textbf{Smooth Pasting Condition} For American put we need the following additional condition $\frac{\partial P}{\partial S} (t, S^P_* (t)) = -1$\\
The smooth pasting condition is not satisfied for all American options. The price $P$ and $\frac{\partial P}{\partial S}$ are continuous\\
If $\frac{\partial P}{\partial S} (t, S^P_* (t))>-1$, then it would be optimal to hold the option until a smaller value than $S_* (t) \Rightarrow S_*(t)$ is not boundary; If $\frac{\partial P}{\partial S} (t, S^P_* (t)) < -1$ then $P(t, S_* (t)) < (K - S_* (t))^+$, arbitrage exist\\

\textbf{Down-and-out call option} with Strike $K$ and barrier $B$ has payoff at $T$ is $(S_t -K )^+$ if $S_t > B$, or $(S_T - K)1_F $, where $F = \{S_T \geq K \min_{t \in [0, T] S_t > B}\}$.\\
\textbf{DI call} has payoff $(S_T - K)1_F, F = \{S_T \geq K, \min_t S_t \leq B\}$ \\
knock-out option + knock-in option = standard option\\
By risk neutral valuation, $c_{di} = e^{-rT} \E^{\Q}[S_T 1_F] - e^{-rT} \E^{\Q} [K 1_F]$, or
$c_{di} (S, B, K, T) = e^{-rT} \E^{\Q} [(S_T - K)1_F ], F = \{S_T \geq K, \min_t S_t \leq B\}$\\

\textbf{Reflection Principle} $m \leq x$, $\prob (W_T \geq x \lvert \tau \leq T) = \prob (\tilde{W}_T \leq 2m-x \lvert \tau \leq T)$\\

\vspace{-5pt}
\begin{flalign*}
    & c_{di}(S, B, K, T) = S_0 \left(\frac{B}{S_0} \right)^{\alpha} N(d_b) - Ke^{-rT} \left(\frac{B}{S_0} \right)^{\alpha-2} N(d_b - \sigma\sqrt{T}) &
\end{flalign*}
where $d_b = \frac{\log \frac{B^2}{S_0 K} + (r + \frac{\sigma^2}{2}) T}{\sigma \sqrt{T}}, \alpha=\frac{2r}{\sigma^2}+1$\\

Let $V_{do} (t, S_t; B, \phi)$ be the price at time $t$ of a down-and-out option with final payoff $\phi(S_T)$ and lower knock-out barrier B. Then, 
\begin{align*}
    V_{do} (t, S_t; B, \alpha \phi + \beta \psi) = \alpha (t, S_t; B, \phi) + \beta (t, S_t; B, \psi)
\end{align*}
\textbf{Put-call Parity} $p_{do}$: price of do put, $b_{do}$: price of do contract with payoff 1, $s_{do}$: price of do contract with payoff $S_T$,
$p_{do} (S,B,T,K) = K b_{do} (S,B,T,K) - s_{do} (S,B,T,K) + c_{do} (S,B,T,K)$\\

\textbf{PDE Approach of Barrier} Consider a knock-out option, prior to knock-out, the option is alive and $V(t, S_t)$ satisfy BS PDE,
$$\frac{\partial V}{\partial t} + \frac{1}{2} \sigma^2 S^2 \frac{\partial^2 V}{\partial S^2} + rS \frac{\partial V}{\partial S} - rV = 0$$
Barrier condition enters through boundary condition and solution domains\\
When barrier is hit, the option becomes worthless: $V(t, B) = 0, \forall t \in [0, T]$\\
Solution domain (for a lower barrier): $[0, T] \times (B, \infty)$\\
If payoff at $T$ is $\phi(S_T)$, then $V(T, S) = \phi(S)$
\begin{Ex}
    Formulate the PDE pricing model for the European down-and-in barrier put option.\\
    Let $V(S, t)$ denote the price of the option, $B$ denote the down barrier. We have $V(S, t) = p(S, t)$ on $\{ S<B \}$, where $p$ is the price of a vanilla put. The price must satisfy the BS PDE $\frac{\partial V}{\partial t} + \frac{1}{2} \sigma^2 S^2 \frac{\partial^2 V}{\partial S^2} + rS \frac{\partial V}{\partial S} - rV = 0$ on $B < S, t \in [0, T)$.\\
    If the stock price reaches the barrier $S=B$ at time $t <T$, the option turns into a put option, i.e. $V(B, t) = p(B, t)$ where $p$ denotes the price of a put.\\
    The upper boundary is $V(S, t) = 0, S\to \infty$.\\
    The terminal condition is: If $S>B$ at $T$, then the option never knocks in and expires worthless. The terminal condition is $V(S, T)=0$
\end{Ex}

\begin{Ex}
    Assume an underlying BS model with $\sigma,r,S_0 >0$ and without dividend payments. Determine the linear complementarity problem for the American up-and-out put option with Barrier $B$ and strike $K$, $K<B, S_0 <B$.\\
    \begin{align*}
        \min \left\{\frac{\partial V}{\partial t} - rS \frac{\partial V}{\partial S} - \frac{1}{2} \sigma^2 S^2 \frac{\partial^2 V}{\partial S^2}+rV, V-(K+S)^+\right\} = 0
    \end{align*}
    Solution domain: $[0, T]\times (0, B)$.\\ Boundary condition: $V(T, S) = (K-S)^+$, Upper boundary: $V(t, B) = 0$
\end{Ex}
\textbf{Asian Option} option whose payoff depends on the average price of the underlying over a certain period\\

\textbf{Geometric Avg} $G_t = \exp \{\frac{1}{t} \int^t_0 \log S_u du \}$, or$X_t = \frac{1}{t} \int^t_0 \log S_u du, G_t = e^{X_t}$\\
Payoff of Asian call with fixed strike $K$: $(e^{X_t} -K)^+$\\
Price of Geometric Asian call at time $t$: $e^{-r(T-t)} \E^{\Q}_t [(G_T - K)^+]$\\
Let $(W_t)_{t \geq 0}$ be a Wiener process, then $\int^T_t (W_u -W_t) du \sim \mathcal{N} (0, \frac{(T-t)^3}{3})$\\

\textbf{Price of Geometric Asian call} $e^{-r(T-t)} \left(G^{t/T}_t S^{(T-t)/t}_{t} e^{\bar{\mu} + \bar{\sigma}^2/2} N(d_2 + \bar{\sigma}) - KN(d_2) \right)$, where $d_2 = \frac{1}{\bar{\sigma}} \left(\frac{t}{T}\log G_t + \frac{T-t}{T} \log S_t - \log K + \bar{\mu} \right)$\\

\textbf{PDE Framework of Asian} $I_t = \int^t_0 f(S_u, u) du, dI_t = f(S_t, t) dt$ is \textbf{path dependent vairable} in price $F(t, S_t, I_t)$ for path-dependent options.\\
Apply multivariate It\^{o}'s Lemma to $F(t) := F(t, S_t, I_T)$, $dF_t$ is,
\begin{align*}
    &\frac{\partial F}{\partial t} dt + \frac{\partial F}{\partial S} dS_t + \frac{\partial F}{\partial I} dI_t + \frac{1}{2} \frac{\partial^2 F}{\partial S^2} (d S_t)^2 + \frac{1}{2} \frac{\partial^2 F}{\partial I^2} (dI_t)^2 + \frac{\partial^2 F}{\partial S \partial I} (dS_t) (dI_t) \\
    &= \left(\frac{\partial F}{\partial t} + \frac{1}{2} \sigma^2 S^2_t \frac{\partial^2 F}{\partial S^2} \right) dt + \frac{\partial F}{\partial S} dS_t + \frac{\partial F}{\partial I} dI_t
\end{align*}
Consider a self-financing and risk-free portfolio $V_t = \gamma_t F_t + \beta_t S_t$, 
\begin{align*}
    dV_t &= \gamma_t dF_t + \beta_t dS_t \\
    &= \gamma_t \left(\frac{\partial F}{\partial t} + \frac{1}{2} \sigma^t S^2_t \frac{\partial^2 F}{\partial S^2} + \frac{\partial F}{\partial I} f(S_t, t) \right) dt + \gamma_t \frac{\partial F}{\partial S} dS_t + \beta_t d S_t
\end{align*}
$\beta_t = - \frac{\partial F}{\partial S} \gamma_t$ to remove $dS_t$, $dV_t = \gamma_t \left(\frac{\partial F}{\partial t} +\frac{1}{2} \sigma^2 S^2_t \frac{\partial^2 F}{\partial S^2} +\frac{\partial F}{\partial I} f(S_t, I) \right) dt$\\
Risk-free condition gives $dV_t = rV_t dt$. $V_t = \gamma_t F_t + \beta_t S_t = \gamma_t F_t - \frac{\partial F}{\partial S}\gamma_t S_t$ gives $\gamma_t = \frac{V_t}{F_t - \frac{\partial F}{\partial S}S_t }$, then,
\begin{align*}
    \frac{\partial F}{\partial t} + r S \frac{\partial F}{\partial S} + \frac{1}{2} \sigma^2 S^2 \frac{\partial^2 F}{\partial S^2} + \frac{\partial F}{\partial I} f(S, t) - rF = 0
\end{align*}


\textbf{Put-call Parity} $l_t = \int^t_0 S_u du$, 
$c(S_t, l_t, t) - p(S_t, l_t, t) = S_t \frac{1- e^{-r (T-t)}}{rT} + e^{-r (T-t)} \left(\frac{l_T}{T} -K \right)$\\

\textbf{Lookback Options} option whose payoff depend on the maximum or the minimum of the underlying stock price over a certain period of time. $m_T = \min_{t\in [0, T]} S_t, M_t = \max_{t\in [0, T]} S_t$\\
\textbf{Running Maximum} Let $f(x)$ be a continuous positive function on $[a, b]$,l then $M = \max_{a\leq x \leq b} f(x) = \lim_{n \to \infty} \left(\int^b_a f(x)^n dx \right)^{1/n}$\\
\vspace{-5pt}

\begin{flalign*}
\hspace{-3pt}
    & M_t = \max_{u\in [0, t]} S_u = \lim_{n \to \infty} \left(\int^t_0 S^n_u du \right)^{\frac{1}{n}} m_t =  \min S_u = \lim_{n} \left(\int^t_0 S^{-n}_u du \right)^{-\frac{1}{n}}&
\end{flalign*}

\textbf{BS PDE for Lookback put with Floating Strike}
\begin{flalign*}
    & \frac{\partial F}{\partial t} + \frac{1}{2} \sigma^2 S^2_t \frac{\partial^2 F}{\partial S^2} + r S_t \frac{\partial F}{ \partial S} - rF = 0, \quad \text{Boundary condition:} &
\end{flalign*}
 $F(T, S, M) = M-S, F(t, 0, M) = e^{-r(T-t)}M, \frac{\partial F}{\partial M} (t, M, M) = 0$\\

 \textbf{Two Asset Option} Assume option has dynamics $d S_{1, t} = \mu_1 S_{1, t} dt + \sigma_1 S_{1, t} d W_{1, t}$, $d S_{2, t} = \mu_2 S_{2, t} dt + \sigma_2 S_{2, t} d W_{2, t}$, $d W_{1, t} d W_{2, t} = \rho dt$, with $d B_t = rB_t dt$\\

 \textbf{2D Girsanov Theorem} Let $(\theta_1, \theta_2)$ be a constant vector, and let $W^{\prob}_{1, t}$, $W^{\prob}_{2, t}$ be independent B.M. under $\prob$. Then exists measure $\Q$ equiv to $\prob$ s.t. 
 \begin{flalign*}
     & d W^{\Q}_{1, t} = d W^{\prob}_{1, t} - \theta_1 dt, d W^{\Q}_{2, t} = d W^{\prob}_{2, t} - \theta_2 dt \text{ are independent B.M. under $\Q$} &
 \end{flalign*}

 \textbf{Price of Exchange Option/Margrabe's Formula} %$V_t = S_{2, t} N(d_1) - S_{1, t} N(d_2)$, $d_1 = \frac{\log \left(\frac{S_{2, t}}{S_{1, t}}\right) + \sigma^2 \frac{T-t}{2}}{\sigma \sqrt{T-t}}$, 
 \vspace{-7pt}
 \begin{flalign*}
     & V_t = S_{2, t} N(d_1) - S_{1, t} N(d_2), d_2 = d_1 - \sigma \sqrt{T-t}, d_1 = \frac{\log \left(\frac{S_{2, t}}{S_{1, t}}\right) + \sigma^2 \frac{T-t}{2}}{\sigma \sqrt{T-t}} &
 \end{flalign*}
 \begin{Ex}
     Suppose we have two traded risky assets $S^{(1)}_t, S^{(2)}_t$, and one risk-free asset with $B_t = e^{rt}$ for $r>0$. The dynamics of the risky assets under the real-world probability measure are given by,
     \begin{align*}
         &dS^{(1)}_t = S^{(1)}_t \mu_1 dt + S^{(1)}_t \sigma_1 dW_{1, t}, dS^{(2)}_t = S^{(2)}_t \mu_2 dt + S^{(2)}_t \sigma_2 dW_{2, t}\\
         &dW_{1, t} d W_{2, t} = \rho dt
     \end{align*}
    where $W_{1, t}, W_{2, t}$ are two correlated Wiener processes, $\mu_1, \mu_2 \in \mathbb{R}$ denote the drifts of the processes, $\sigma_1, \sigma_2>0$ denote the volatilities of the first and second asset and $\rho \in [-1, 1]$. Find the arbitrage-free price at time $0$ of a financial derivative with payoff $\phi(S^{(1)}_T, S^{(2)}_T)$ at $T$, where $$\phi(S^{(1)}_T, S^{(2)}_T) := \max\left\{ aS^{(1)}_T, bS^{(2)}_T \right\}, a,b>0$$
    For all $x>0$ it holds that $\max\{ x, 1\} = 1+ \max \{x-1, 0\}$, then,
    \begin{align*}
        &\frac{\phi (S^{(1)}_T, S^{(2)}_T)}{S^{(1)}_T} = \frac{\max\left\{ aS^{(1)}_T, bS^{(2)}_T \right\}}{S^{(1)}_T} = \max \left\{\frac{bS^{(2)}_T}{S^{(1)}_T}, a \right\} = a \max \left\{\frac{bS^{(2)}_T}{aS^{(1)}_T}, 1 \right\}\\
        &\Rightarrow a+a \max \left\{\frac{bS^{(2)}_T}{aS^{(1)}_T}-1, 0 \right\} = a+a \max \left\{\frac{b}{a}Z-1, 0 \right\} = a+ \max \{bZ-a, 0 \}\\
        &\Rightarrow a+ (bZ-a)^+
    \end{align*}
    where $Z := S^{(2)}_T / S^{(1)}_T$. Consider the measure $\mathbb{Q}^*$ which takes into account $S^{(a)}_T$ as a numeraire. Under $\Q^*$ we have $Z = \frac{S^{(2)}_T}{S^{(1)}_T} e^U, U\sim \mathcal{N} (-\frac{1}{2}\sigma^2 T, \sigma^2 T), \sigma = \sqrt{\sigma_1^2 - 2\rho \sigma_1 \sigma_2 + \sigma^2_2}$. Then we compute 
    \begin{align*}
        V_0 = \E_{\Q} [e^{-rT}\phi] = S^{(1)}_T \E_{\Q^*}[a+ (bZ-a)^+] = S^{(1)}_T (a+P)
    \end{align*}
    where $P$ is the price of call option with strike $a$ in BS model with $r=0, S_0 = bS^{(2)}_T / S^{(1)}_T$, i.e. $P = b \frac{S^{(2)}_T}{S^{(1)}_T} N(d_1) - aN(d_2)$,
    \begin{align*}
        d_2 = \frac{1}{\sigma \sqrt{T}} \left(\log \left( \frac{bS^{(2)}_T}{aS^{(1)}_T}\right) -\frac{1}{2}\sigma^2 T \right), d_1 = d_1 + \sigma\sqrt{T}
    \end{align*}

If $\phi (S^{(1)}_T, S^{(2)}_T) := S^{(1)}_T \mathds{1}_{S^{(1)}_T \leq KS^{(1)}_T}$,\\
Under change of numeraire when using $S^{(1)}_T$ as a numeraire, 
\begin{align*}
    \frac{S^{(2)}_T}{S^{(1)}_T} \sim \frac{S^{(2)}_T}{S^{(1)}_T} \exp \left\{\frac{1}{2}\sigma^2 T + \sigma \sqrt{T}X \right\}, X \sim \mathcal{N}(0, 1)
\end{align*}
under some measure $\Q^*$ and $\sigma = \sqrt{\sigma_1^2 - 2\rho \sigma_1 \sigma_2 + \sigma^2_2}$. Let $\Q$ denote the risk-neutral measure with $B_t$ as a numeraire. Then, by a change of numeraire,
\begin{align*}
    \E_{\Q} \left[e^{-rT}S^{(1)}_T \mathds{1}_{S^{(2)}_T < K S^{(1)}_T} \right] &= S^{(1)}_T \E_{\Q^*} \left[\frac{S^{(2)}_T}{S^{(1)}_T} \mathds{1}_{S^{(2)}_T < K S^{(1)}_T} \right]\\
    &=S^{(1)}_T \Q^* \left(\frac{S^{(2)}_T}{S^{(1)}_T} \exp \left\{\frac{1}{2}\sigma^2 T + \sigma \sqrt{T}X \right\} \leq K \right)
\end{align*}
\begin{align*}
    S^{(1)}_T \Q^* \left(X \leq \frac{\log \left(K \frac{S^{(1)}_T}{S^{(2)}_T} \right)+ \frac{1}{2}\sigma^2 T }{\sigma \sqrt{T}} \right)=S^{(1)}_T N \left(\frac{\log \left(K\frac{S^{(1)}_T}{S^{(2)}_T} \right) + \frac{1}{2}\sigma^2 T}{\sigma \sqrt{T}} \right)
\end{align*}
\end{Ex}

 
 
\end{multicols*}
\end{document}
